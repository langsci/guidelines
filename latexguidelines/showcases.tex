\chapter{Showcases}

There is a huge amount of packages that can be used for various purposes. \citet{MG2013a} is a good
reference book. This section discusses some aspects of some packages that are relevant for
linguistics. Every \LaTeX\xspace package comes with a documentation and users should consult these
documentations, too. The purpose of this section is to point users to the packages that we think
serve their purpose best and that are compatible with other packages and the \lsp classes.

\section{Glossed examples}

Glossed\is{glossing|(}\ispackageb{lsp-gb4e} examples are typeset with a modified version of the \texttt{gb4e}\ispackage{gb4e} package by Craig
Thiersch\ia{Thiersch, Craig}. The modified package is called \texttt{langsci-gb4e}. 
% % It is contained in the styles directory that is delivered with the \lsp \LaTeX\xspace classes. It differs from the original package in loading a version of \texttt{gloss} that was modified by Alexis Dimitriadis\ia{Alexis Dimitriadis} in order to be compatible with \texttt{jambox} (see Section~\ref{sec-jambox}).

Simple examples like \REF{ex:showcases:simple} can be typeset as shown below.
\ea\label{ex:showcases:simple} 
\gll Der Mann schläft.\\
     the man  sleeps\\
\glt `The man sleeps.'
\z
\begin{verbatim}
\ea
\gll  Der Mann schläft.\\
      the man  sleeps\\
\glt `The man sleeps.'
\z
\end{verbatim}

Grammaticality judgments can be added in brackets. Note that in this case, braces have to be used around the rest of the example
\ea[*]{\label{ex:showcases:ungrammatical} 
\gll Der Mann schlafen.\\
     the man  sleep\\
\glt `(The man sleeps.)'
}
\z
 
\newpage
\begin{verbatim}
\ea[*]{
  \gll  Der Mann schlafen.\\
        the man  sleep\\
  \glt `(The man sleeps.)'
  }
\z
\end{verbatim} 


Lists of examples can be typeset with nested \verb+\ea+  and \verb+\z+  respectively. The example in
\REF{ex:showcases:list} shows how the sentences can be aligned properly. Note that the first example in a list gets \verb+\ea+, the subsequent ones get \verb+\ex+. Also note the empty grammticality judgment for the first example in order to align it with the second example, which has a~*.


\ea  \label{ex:showcases:list}
\ea[]{ 
\gll Ich glaube dem Linguisten nicht, 
     einen Nobelpreis gewonnen zu haben.\\
     I believe the linguist not 
     a Nobel.prize won to have\\
\glt `I don't believe linguist's claim 
     that he won a Nobel prize.'
}
\ex[*]{
\gll Dem Linguisten einen Nobelpreis  glaube  
     ich nicht gewonnen zu haben.\\
     the linguist   a     Nobel.prize believe 
     I   not   won      to have\\
}
\z
\z  

\begin{verbatim}

\ea 
 \ea[]{
 \gll Ich glaube dem Linguisten nicht, 
      einen Nobelpreis gewonnen zu haben.\\
      I believe the linguist not 
      a Nobel.prize won to have\\
 \glt `I don't believe linguist's claim 
      that he won a Nobel prize.'
}
\end{verbatim}
\newpage
\begin{verbatim}
 \ex[*]{
 \gll Dem Linguisten einen Nobelpreis  glaube  
      ich nicht gewonnen zu haben.\\
      the linguist   a     Nobel.prize believe 
      I   not   won      to have\\
 }
 \z
\z
\end{verbatim}

If you want to add a footnote\is{footnote|(} that provides the source of an example as in \REF{ex:showcases:footnote}, you can do
this as follows:
\ea\label{ex:showcases:footnote}
\gll Piloten         fik frataget    sit certifikat\footnotemark\\
     pilot.\textsc{def} got deprived.of his license\\
\footnotetext{KorpusDK.}
\glt `The pilot was deprived of his license to fly.'
\z 
\begin{verbatim}
\ea
\gll Piloten fik frataget sit certifikat{\footnotemark}\\
     pilot.\textsc{def} got deprived.of his license\\
\footnotetext{KorpusDK.}
\glt `The pilot was deprived of his license to fly.'
\z 
\end{verbatim}
Please call the \verb+\footnotetext+ command before the translation, since otherwise the
footnotetext may be typeset on a page that is different from the one where the footnotemark is set.\is{footnote|)}

% For the typesetting of an additional line with the original script, one may use \verb+\glll+ rather
% than \verb+\gll+. \REF{ex:shown:chinese} shows a Chinese example:
% \ea
% \label{ex:shown:chinese}
% \glll 狗       叫     了。\\
%       gou3     jiao4   le\\
%       dog      bark    \textsc{asp/crs}\\
% \glt `The dog is barking.'/`The dogs are barking.'
% \z
% 
% \begin{verbatim}
% \ea
% \glll 狗       叫     了。\\
%       gou3     jiao4   le\\
%       dog      bark    \textsc{asp/crs}\\
% \glt `The dog is barking.'/`The dogs are barking.'
% \z
% \end{verbatim}

% You can use up to \verb+\glllllll+ if you need additional lines.

In some subdisciplines of linguistics (e.\,g.\ typology) the examples are written in italics.
This is done automatically according to the series you publish in. 
% of course this is not true for the code above ... 

If the series decides to use italics, it has to be ensured that structural markup like brackets are not typeset in italics. Use \verb+\ob+ for opening brackets and \verb+\cb+ for closing brackets. \verb+\op+ and \verb+\cp+ provide the same for parens.
\begin{verbatim}
\ea
\gll ein {\ob}interessantes  Beispiel{\cb}\\
     an   interesting        example\\
\glt `an interesting example'
\z 
\end{verbatim}
\ea
\def\exfont{\normalsize\itshape}
\gll ein {\ob}interessantes       Beispiel{\cb}\\
     an   interesting example\\
\glt `an interesting example'
\z 
\is{glossing|)}\ispackagee{lsp-gb4e}

In order to align the gloss with the beginning of the source word, and not with the bracket, you can use \verb+{\db}+ (dummy bracket).

\begin{verbatim}
\ea
\gll ein {\ob}interessantes  Beispiel{\cb}\\
     an  {\db}interesting        example\\
\glt `an interesting example'
\z 
\end{verbatim}

\ea
\def\exfont{\normalsize\itshape}
\gll ein {\ob}interessantes       Beispiel{\cb}\\
     an  {\db}interesting example\\
\glt `an interesting example'
\z 
 


In typological series examples often come with the language name and references. The examples on
page~\pageref{ex-typology} are typeset as follows:
\begin{verbatim}
\ea
  \langinfo{Mising}{Sino-Tibetan}{\citealt[69]{Prasad91a}}\\
  \gll azɔ́në dɔ́luŋ\\
      small village\\ 
  \glt `a small village' 
\z
\end{verbatim}
\ea
  \langinfo{Mising}{Sino-Tibetan}{\citealt[69]{Prasad91a}}\\
  \gll azɔ́në dɔ́luŋ\\
      small village\\ 
  \glt `a small village' 
\z

\newpage

\begin{verbatim}
\ea 
  \ea
    \langinfo{Apatani}{Sino-Tibetan}{\citealt[23]{Abraham85a}}\\
    \gll aki atu\\ 
         dog small\\ 
    \glt ‘the small dog’ 
  \ex
  \langinfo{Temiar}{Austroasiatic}{\citealt[155]{Benjamin76a}}\\ 
    \gll dēk mənūʔ\\
         house big\\
    \glt ‘big house’ 
  \z
\z
\end{verbatim}

\ea 
  \ea
    \langinfo{Apatani}{Sino-Tibetan}{\citealt[23]{Abraham85a}}\\
    \gll aki atu\\ 
	dog small\\ 
    \glt ‘the small dog’ 
  \ex
  \langinfo{Temiar}{Austroasiatic}{\citealt[155]{Benjamin76a}}\\ 
    \gll dēk mənūʔ\\
	house big\\
    \glt ‘big house’ 
  \z
\z

\def\exfont{\normalsize\upshape}

\section{Movement arrows}
\texttt{langscibook} provides an alternative to \texttt{gb4e}'s movement arrows. This module is powered by TikZ and designed to work in example environments. A movement arrow consists of a pair of \verb+\ConnectTail+ and \verb+\ConnectHead+ commands, where the tail \emph{always} precedes the head. The syntax is as follows:\\

\noindent\verb+\ConnectTail{<Text>}[<Group>]+ initialises a group. In more complex situations, naming the groups is required for correct display. Please \emph{use only word characters} (A--z, 0--9), but no spaces, brackets, or other special characters inside \verb+[Group]+.\\ 

\noindent\verb+\ConnectHead<*>[<Spacing>]{<Text>}[<Group>]+ gives a pairing head to a previously defined tail. The head also constructs the arrow, and as two additional arguments. When used in its normal form, \verb+\ConnectHead+ creates a left-to-right arrow. When the starred version \verb+\ConnectHead*+ is used, the resulting arrow is right-to-left. The optional \verb+[Spacing]+ (standard is \verb+1ex+) gives the space between the text's baseline and the arrow. \verb+{Text}+ and \verb+[Group]+ work as in the tail. If you have specified a group in the tail, please don't forget to allocate the head to the same group.

\subsection{Simple examples}
In the standard form, the procedure expects that movement arrows do not cross each other:

\ea \ConnectTail{This} is \ConnectHead{an} \ConnectTail{example} of \ConnectHead*{movement}.\\
\begin{lstlisting}
\ea 
    \ConnectTail{This} is \ConnectHead{an} \ConnectTail{example} of \ConnectHead*{movement}. 
\z
\end{lstlisting}
\z

\subsection{Crossing movements}
In case your movement arrows cross each other, please specify names and distances for all but one group:

\ea \ConnectTail{This}[A] is \ConnectTail{an} \ConnectHead{example} of \ConnectHead*[2ex]{movement}[A].\\
\begin{lstlisting}
\ea 
    \ConnectTail{This}[A] is \ConnectTail{an} \ConnectHead{example} of \ConnectHead*[2ex]{movement}[A]. 
\z
\end{lstlisting}
\z
Note that you have to assign the \verb+[Group]+ to both tail and head, but \verb+*+ and \verb+[Spacing]+ are only valid as head arguments.

% \subsection{Correcting spacing}
% As arrows are further distanced from their respective text, they may interfere with following lines. It's advisable to adjust the line spacing in these cases. If you use simple example environments, a \verb+\doublespacing+ will suffice:\largerpage
% 
% \begin{exe}\doublespacing
% \ex\ConnectTail{This}[A] is \ConnectTail{an} \ConnectHead{example} of \ConnectHead*[2ex]{movement}[A].
% \ex Another Sentence
% 
% \largerpage[-1]
% \singlespacing
% \begin{lstlisting}
% \begin{exe}
%     \doublespacing
%     \ex\ConnectTail{This}[A] is \ConnectTail{an} \ConnectHead{example} of \ConnectHead*[2ex]{movement}[A].
%     \ex Another Sentence
% \end{exe}
% \end{lstlisting}
% \end{exe}
% Make sure that \verb+\doublespacing+ is given \emph{inside} the example environment, so that it will only be applied to this example.
% 
% Using movement arrows with glossing and in trees is still under development.

\section{\texttt{jambox}}
\label{sec-jambox}


The\ispackageb{jam\-box} package \texttt{jambox} by Alexis Dimitriadis\ia{Dimitriadis, Alexis} can be used to provide information about the language of an example or
about a certain other aspect to be highlighted.\il{Maltese}

\settowidth\jamwidth{VSO}
\eal
\ex[]{
\label{ex-ingrid-kielet-ilmazzita}
\gll Ingrid kiel-et il-mazzit-a.\\
     Ingrid eat-{\scshape 3sg.f} {\scshape def}-black.pudding-{\scshape sg.f}\\ \jambox{(SVO)}
\glt `Ingrid ate black pudding.'
}
\ex[]{
Kielet ilmazzita Ingrid. \jambox{(VOS)}
}
\ex[*]{
Kielet Ingrid ilmazzita. \jambox{(VSO)}
}
\ex[]{\label{ex-sov}
Ingrid ilmazzita kielet. \jambox{(SOV)}
}
\ex[]{\label{ex-osv}
Ilmazzita Ingrid kielet. \jambox{(OSV)}
}
\ex[]{
Ilmazzita kielet Ingrid. \jambox{(OVS)}
}
\zl

The call of \verb+\jambox+ has to follow the linebreak after the gloss:
\begin{verbatim}
\ex[]{
\label{ex-ingrid-kielet-ilmazzita}
\gll Ingrid kiel-et il-mazzit-a.\\
     Ingrid eat-3fsg def-black.pudding-fsg\\ \jambox{(SVO)}
\glt `Ingrid ate black pudding.'
}
\end{verbatim}
The distance from the right margin can be specified by passing the largest object to be placed in a
jambox to \verb+\settowidth+:

\ea 
\settowidth\jamwidth{(German)}
  \ea The man reads the book.    \jambox{(English)}
  \ex Manden læser bogen.        \jambox{(Danish)}
  \ex Der Mann liest das Buch.   \jambox{(German)}
  \z 
\z

\begin{verbatim}
\ea 
\settowidth\jamwidth{(German)}
  \ea The man reads the book.    \jambox{(English)}
  \ex Manden læser bogen.        \jambox{(Danish)}
  \ex Der Mann liest das Buch.   \jambox{(German)}
  \z
\z
\end{verbatim}
\ispackagee{jam\-box}



\section{Trees: \texttt{forest}}
Linguistic trees can be typeset with the \verb+forest+ package. An example is given below.

\begin{verbatim}
\begin{forest}
  [VP
    [DP[John]]
    [V’
      [V[sent]]
      [DP[Mary]]
      [DP[D[a]][NP[letter]]]
    ]
  ]
\end{forest}
\end{verbatim}
\begin{forest}
[VP
[DP[John]]
[V’
[V[sent]]
[DP[Mary]]
[DP[D[a]][NP[letter]]]
]
]
\end{forest}
% 
% A more complicated example, showing the power of the \verb+forest+ package is given below
% 
% \begin{verbatim}
%  \begin{forest}
% myGP1/.style={
% GP1,
% delay={where tier={x}{
% for children={content=\textipa{##1}}}{}},
% tikz={\draw[dotted](.south)--
% (!1.north west)--(!l.north east)--cycle;},
% for children={l+=5mm,no edge}
% }
% [VP[DP[John,tier=word,myGP1
% [O[x[dZ]]]
% [R[N[x[6]]]]
% [O[x[n]]]
% [R[N[x]]]
% ]][V’[V[loves,tier=word,myGP1
% [O[x[l]]]
% [R[N[x[a]]]]
% [O[x[v]]]
% [R[N[x]]]
% [O[x[z]]]
% [R[N[x]]]
% ]][DP[Mary,tier=word,myGP1
% [O[x[m]]]
% [R[N[x[e]]]]
% [O[x[r]]]
% [R[N[x[i]]]]
% ]]]]
% \end{forest}%
% \end{verbatim}
% \begin{forest}
% myGP1/.style={
% GP1,
% delay={where tier={x}{
% for children={content=\textipa{##1}}}{}},
% tikz={\draw[dotted](.south)--
% (!1.north west)--(!l.north east)--cycle;},
% for children={l+=5mm,no edge}
% }
% [VP[DP[John,tier=word,myGP1
% [O[x[dZ]]]
% [R[N[x[6]]]]
% [O[x[n]]]
% [R[N[x]]]
% ]][V’[V[loves,tier=word,myGP1
% [O[x[l]]]
% [R[N[x[a]]]]
% [O[x[v]]]
% [R[N[x]]]
% [O[x[z]]]
% [R[N[x]]]
% ]][DP[Mary,tier=word,myGP1
% [O[x[m]]]
% [R[N[x[e]]]]
% [O[x[r]]]
% [R[N[x[i]]]]
% ]]]]
% \end{forest}%

\section{Diagrams}

For easy diagrams, we provide the command \verb+\barplot+ (see \figref{fig:barplot}). More complicated programs can be drawn with Ti\textit{k}Z as well, please contact support. For diagrams consisting only of lines, curves, boxes, and symbols, please provide the data, we will draw them in Ti\textit{k}Z. For heatmaps and the like, please provide high resolution graphics files. 



\begin{verbatim}
\begin{figure} 
  \barplot{Person}{\%}{P01,P02,P03}{
	      (P01,19.47733441) 
	      (P02,04.99311069) 
	      (P03,01.22486586)
  }
  \caption{Ratio of fixation time in the caption area in
 relation to fixation time to the whole screen}
\end{figure}
\end{verbatim}

\begin{figure} 
  \barplot{Person}{\%}{P01,P02,P03}{
	      (P01,19.47733441) 
	      (P02,04.99311069) 
	      (P03,01.22486586)
  }
  \caption{Ratio of fixation time in the caption area in relation to fixation time to the whole screen}
  \label{fig:barplot}
\end{figure}


\section{Subfloats and row placement of floats}
To place (sub-)figures and (sub-)tables beside each other, please use the \texttt{subcaption} and \texttt{floatrow}. An example with subfigures is displayed in \figref{fig:showcases:subfigures}, and an example of two figures in row placement in Figures~\ref{fig:showcases:rowA}--\ref{fig:showcases:rowB}. The example is taken from \textcite{Müller2021}.

\begin{verbatim}
\begin{figure}\hfill%
	\begin{subfigure}[b]{.3\textwidth}\centering
		\begin{forest}
			[V$'$ [V [read]] [blub$_i$]]
		\end{forest}
			\caption{Trace\label{fig-a}}
	\end{subfigure}\hfill%
	\begin{subfigure}[b]{.3\textwidth}\centering
		\begin{forest}
			[V$'$ [V [read]] [NP[bla$_i$]]]
		\end{forest}
		\caption{XP with empty daughter}
	\end{subfigure}\hfill%
	\begin{subfigure}[b]{.3\textwidth}\centering
		\begin{forest}
			[V$'$ [V [read]] [NP$_i$ [bla]]]
		\end{forest}
		\caption{Mix of a and b\label{fig-b}}
		\end{subfigure}\hfill
	\caption{Alternative ways of depicting movement: the moved constituent can be represented by a trace
or by an XP dominating a trace}\label{fig:showcases:subfigures}
\end{figure}
\end{verbatim}

\begin{figure}\hfill%
	\begin{subfigure}[b]{.3\textwidth}\centering
		\begin{forest}
			[V$'$ [V [read]] [blub$_i$]]
		\end{forest}
			\caption{Trace\label{fig-a}}
	\end{subfigure}\hfill%
	\begin{subfigure}[b]{.3\textwidth}\centering
		\begin{forest}
			[V$'$ [V [read]] [NP[bla$_i$]]]
		\end{forest}
		\caption{XP with empty daughter}
	\end{subfigure}\hfill%
	\begin{subfigure}[b]{.3\textwidth}\centering
		\begin{forest}
			[V$'$ [V [read]] [NP$_i$ [bla]]]
		\end{forest}
		\caption{Mix of a and b\label{fig-b}}
		\end{subfigure}\hfill
	\caption{Alternative ways of depicting movement: the moved constituent can be represented by a trace
or by an XP dominating a trace}\label{fig:showcases:subfigures}
\end{figure}

\begin{verbatim}
\begin{figure}
\begin{floatrow}
  \captionsetup{margin=.05\linewidth}
  \ffigbox[.5\linewidth]
          {...}
          {\caption{Figure A\label{fig:showcases:rowA}}}
  \ffigbox[.5\linewidth]
          {...}
          {\caption{Figure B\label{fig:showcases:rowA}}}
\end{floatrow}
\end{figure}
\end{verbatim}

\begin{figure}
\begin{floatrow}
  \captionsetup{margin=.05\linewidth}
  \ffigbox[.5\linewidth]
          {...}
          {\caption{Figure A\label{fig:showcases:rowA}}}
  \ffigbox[.5\linewidth]
          {...}
          {\caption{Figure B\label{fig:showcases:rowA}}}
\end{floatrow}
\end{figure}

% \section{DRSes: \texttt{drs}}
% 
% DRSes\ispackageb{drs} can be typeset using the \texttt{drs} package by Alexis Dimitriadis\ia{Dimitriadis, Alexis}. There are various commands that let you typeset simple DRSes, ones with implications
% and DRSes with quantifiers. Some examples from the manual are given below:
% 
% \bigskip
% 
% \drs{x y}{Jones(x) \\ Ulysses(y) \\ x owns y}
% 
% \begin{verbatim}
% \drs{x y}{Jones(x) \\ Ulysses(y) \\ x owns y}
% \end{verbatim}
% 
% 
% \drs{x}{Jones(x) \\
%       \ifdrs{y}{donkey(y)\\x owns y}
%             {z w}{z = x\\ w = y\\ z feeds w}}
% 
% \begin{verbatim}
% \drs{x}{Jones(x) \\
%       \ifdrs{y}{donkey(y)\\x owns y}
%             {z w}{z = x\\ w = y\\ z feeds w}}
% \end{verbatim}
% 
% \drs{X}{ the lawyers(X) \\
%         \qdrs{x}{x $\in$ X}
%              {every}{x}
%              {y}{secretary(y) \\ x hired y}}
% 
% \begin{verbatim}
% \drs{X}{ the lawyers(X) \\
%         \qdrs{x}{x $\in$ X}
%              {every}{x}
%              {y}{secretary(y) \\ x hired y}}
% \end{verbatim}
% \ispackagee{drs}
% 
% %%%%%%%%%%%%%%%%%%%%%%%%%%%%%%%%%%%%%%%%%%%%%%%%%%%%%%%%%%%%%%%%%%%%%%%%%%%%%%%%%%%%%%%%%%%%%%%%%%%
% 
% \section{AVMs}
% 
% The package for typesetting AVMs that is most widely used is the package \texttt{avm}\ispackage{avm}
% by Chris Manning\ia{Manning, Chris}. 
% 
% %% This package is described in the following
% %% subsection. Section~\ref{sec-lsp-avm} describes AVM macros that were put together by Markus Duda\ia{Markus Duda}.
% 
% %% \subsection{\texttt{avm}}
% 
% \REF{ex:showcases:avm-avm}\ispackageb{avm} shows an example of an AVM typeset with the \texttt{avm} package:
% \ea
% \label{ex:showcases:avm-avm}
% \begin{avm}
% \[phon   & \< {\itshape porcupine\/} \>\\
%   feat-a & \@{10} \[feat-aa & type-aa\\
%                     feat-ab & \< \[ synsem|loc|cat|head & type-aba\\
%                                     feat-abc \tpv{type-abc} 
%                                   \],
%                                   \textup{NP} \>\\
%                     \tp{type-a}
%                   \]\\
%  feat-b & \@{10} type-b\\ 
%  \tp{some-type}
% \]
% \end{avm}
% \z
% 
% 
% 
% \begin{verbatim}
% \begin{avm}
% \[phon   & \< {\itshape porcupine\/} \>\\
%   feat-a & \@{10} \[feat-aa & type-aa\\
%                     feat-ab & \< \[ synsem|loc|cat|head & type-aba\\
%                                     feat-abc \tpv{type-abc} 
%                                   \],
%                                   \textup{NP} \>\\
%                     \tp{type-a}
%                   \]\\
%  feat-b & \@{10} type-b\\ 
%  \tp{some-type}
% \]
% \end{avm}
% \end{verbatim}
% %
% The command \verb+\tp+ is defined as follows (the code is taken from Detmar
% Meurers'\ia{Meurers, Detmar} \texttt{avm+}\ispackage{avm+}):
% \begin{verbatim}
% % command to fontify the type values of an avm 
% \newcommand{\tpv}[1]{{\avmjvalfont #1}}
% 
% % command to fontify the type of an avm and avmspan it
% \newcommand{\tp}[1]{\avmspan{\tpv{#1}}}
% \end{verbatim}
% 
% A more complex example is given in \REF{ex:showcases:avm-complicated}:
% \ea\label{ex:showcases:avm-complicated} 
%   \begin{avm}
%     {\itshape word\/} $\rightarrow$
%     \[ morphs & $\@{e_1}\bigcirc\cdots\bigcirc\@{e_n}$\\
%        morsyn & \@0 $(\@{m_1}\uplus\cdots\uplus\@{m_n})$\\
%        rules  & \< \[ morphs & \@{e_1}\\
%                       mud & \@{m_1}\\ 
%                       morsyn & \@0\], \ldots ,
%                     \[morphs & \@{e_n}\\
%                       mud    & \@{m_n}\\ 
%                       morsyn & \@0\] \>
%     \]
%   \end{avm}
% \z
% 
% 
% The code is given below:
% \begin{verbatim}
%   \begin{avm}
%     {\itshape word\/} $\rightarrow$
%     \[ morphs & $\@{e_1}\bigcirc\cdots\bigcirc\@{e_n}$\\
%        morsyn & \@0 $(\@{m_1}\uplus\cdots\uplus\@{m_n})$\\
%        rules  & \< \[ morphs & \@{e_1}\\
%                       mud & \@{m_1}\\ 
%                       morsyn & \@0\], \ldots,
%                     \[morphs & \@{e_n}\\
%                       mud    & \@{m_n}\\ 
%                       morsyn & \@0\] \>
%     \]
%   \end{avm}
% \end{verbatim}
% With the \texttt{avm} package it is possible to use brackets as they are used in AVMs.
% 
% The package has a good documentation and we will not repeat all the details here.
% \ispackagee{avm}
% 
% %% \subsection{\texttt{lsp-avm}}
% %% \label{sec-lsp-avm}
% 
% %% An alternative way to typeset AVMs is provided in the \lsp style file \texttt{lsp-avm} that contains
% %% code by Markus Duda\ia{Markus Duda} with some adaptions by Stefan Müller\ia{Stefan
% %%   M{\"u}ller}. The AVM in (\ref{ex-avm-avm}) is typeset as follows:
% 
% %% \ea
% %% \ms[some-type]{
% %%   phon   & \phonliste{ porcupine } \\
% %%   feat-a & \ibox{10} \ms[type-a]{ feat-aa & type-aa\\
% %%                                   feat-ab & \liste{ \onems{ synsem|loc|cat|head \type{type-aba}\\
% %%                                                             feat-abc \type{type-abc} },
% %%                                   NP }\\ }\\
% %%  feat-b & \ibox{10} type-b\\ 
% %% }
% %% \z
% 
% %% %\ea
% %% \begin{verbatim}
% %% {\itshape word\/} $\rightarrow$
% %%     \ms{ morphs & $\ibox{e_1}\bigcirc\cdots\bigcirc\ibox{e_n}$\\
% %%          morsyn & \ibox{0} $(\ibox{m_1}\uplus\cdots\uplus\ibox{m_n})$\\
% %%          rules  & \liste{ \ms{ morphs & \ibox{e_1}\\
% %%                                mud & \ibox{m_1}\\ 
% %%                                morsyn & \ibox{0} }, \ldots ,
% %%                           \ms{ morphs & \ibox{e_n}\\
% %%                                mud    & \ibox{m_n}\\ 
% %%                                morsyn & \ibox{0} } }
% %%       }
% %% \end{verbatim}
% 
% 
% % \section{OT tableaux}
% % 
% % 
% % This\is{Optimality Theory|(}\is{tabular} section just provides a simple example of how Optimality Tableaux can be typeset.
% % 
% % 
% % 
% % \begin{verbatim}
% % \begin{tabular}[t]{r|c|c|c|}
% % \cline{2-4}
% %       & /qi/  & qi    & qi         \\
% % \LCC 
% %       &       &       & \lightgray \\ \cline{2-4}
% % \hand & [qi]  &       & *          \\ \cline{2-4}
% %       & [*qi] & *!    &            \\ \cline{2-4}
% % \ECC
% % \end{tabular}
% % \end{verbatim}
% % 
% % \begin{tabular}[t]{r|c|c|c|}
% % \cline{2-4}
% %       & /qi/  & qi    & qi         \\
% % \LCC 
% %       &       &       & \lightgray \\ \cline{2-4}
% % \hand & [qi]  &       & *          \\ \cline{2-4}
% %       & [*qi] & *!    &            \\ \cline{2-4}
% % \ECC
% % \end{tabular}
% % 
% % % \section{Conversation transcripts} 
% % % To be completed.
% % 
% % % \section{Font issues and right to left scripts}
% % % 
% % % Since\is{font|(} we are using \xelatex, all fonts that are installed in the cannonical font directories can be
% % % used. We are using the font Linux Libertine, which is unicode-based and contains a lot of
% % % the characters linguists want to use.
% % % 
% % % \subsection{Chinese}
% % % \label{sec-Chinese}
% % % 
% % % You can enter Chinese\il{Chinese}\is{Chinese Characters} characters directly and mix them with ASCII text without any further markup
% % % provided you load the \texttt{xeCJK}\ispackage{xeCJK} package. We already saw an example in (\ref{ex-chinese}) on
% % % page~\pageref{ex-chinese}. In order to type Chinese text, one has to load the \texttt{xeCJK} package
% % % with the option \indentfirst set to \false and select an appropriate font:
% % % \begin{verbatim}
% % % \usepackage[indentfirst=false]{xeCJK}
% % % \setCJKmainfont{SimSun}
% % % \end{verbatim}
% % % 
% % % 
% % % \subsection{Arabic script}
% % % 
% % % Arabic script\is{Arabic Script}\il{Persian} is the most challenging script for typesetting since it is written from right to left
% % % and contains ligatures. If you load the \texttt{bidi} package, you can mix right to left and left to
% % % right text.\footnote{
% % %   Please have a look at the source code. The verbatim environment has difficulties to display Arabic
% % %   text and hence the call to \texttt{$\backslash$PRL} comes out scrambled.
% % % }
% % % 
% % % \ea
% % % % \PRL{او مرد را دوست نخواهد داشت.}\\
% % %  \gll U      mard rā        dust   naxāhad        dāšt.\\
% % %       He/she man  {\scshape dom} friend {\scshape neg}.want have\\
% % % \glt `He/she will not love the man.'
% % % \z
% % % 
% % % %\begin{rtlverbatim}
% % % %\usepackage{fontspec}
% % % \begin{verbatim}
% % % \newfontfamily\Parsifont[Script=Arabic]{XB Niloofar}
% % % \usepackage{bidi}
% % % \newcommand{\PRL}[1]{\RL{\Parsifont #1}}
% % % 
% % % \ea
% % % \PRL{او مرد را دوست نخواهد داشت.}\\
% % % \gll U      mard rā       dust   naxāhad        dāšt.\\
% % %      He/she man {\scshape dom} friend {\scshape neg}.want have\\
% % % \glt `He/she will not love the man.'
% % % \z
% % % \end{verbatim}
% % % %\end{rtlverbatim}
% % % 
% % % \subsection{Hebrew}
% % % 
% % % Hebrew\il{Hebrew|(} is also written from right to left. The characters are part of Linux Libertine, so no extra
% % % font has to be loaded to set examples like \REF{ex:latex:hebrew}:
% % % \ea\label{ex:latex:hebrew}
% % % % \RL{האישה קוראת ספר.}\\
% % % \gll   ha-'iša          qore't                            sefer.\\
% % %        {\scshape def}-woman  read.{\scshape pres}.{\scshape f}.{\scshape sg}  book\\
% % % \glt `The woman is reading a book.'
% % % \z
% % % % \begin{fitverb}
% % % \ea
% % % % \RL{האישה קוראת ספר.}\\
% % % \gll   ha-'iša          qore't                            sefer.\\
% % %        {\scshape def}-woman  read.{\scshape pres}.{\scshape f}.{\scshape sg}  book\\
% % % \glt `The woman is reading a book.'
% % % \z
% % % % \end{fitverb}
% % % \il{Hebrew|)}
% % % 
% % % \subsection{IPA symbols}
% % % 
% % % The\is{IPA symbols|(} IPA symbols are part of the Linux Libertine font and hence can be entered into the document
% % % directly. The IPA unicode symbols can be created online at
% % % \url{http://ipa.typeit.org/full/}. \REF{ex:showcases:IPA} shows some examples:
% % % \ea\label{ex:showcases:IPA}
% % % ɓ ɐ ʁ ɾ ɻ ʃ ʂ θ~  t͡ʃ~  t͡s  ʈ ʊ ʊ̈ ʉ ʌ ʋ ʍ ɯ ɰ χ ʎ ɣ ʏ ɤ ʒ ʐ ʑ ʔ ʕ ʢ ʡ ɑ̃ ɔ ˧ ˨ ˩ ˩˥ ˥˩ ˦˥
% % % \z
% % % % ⱱ does not work  
% % % If you find symbols that are not covered by the font, please use the \texttt{tipa} package.
% % % \is{font|)}\is{IPA symbols|)}
% % %  
% % % % \section{Bells and whistles}
% % % % 
% % % % \section{\texttt{varioref}}
% % % % 
% % % % \texttt{varioref}\ispackageb{varioref} is loaded by the \lsp class file. You can use \vref \iscommand{vref} to refer to floating
% % % % objects like figures and tables. \LaTeX\xspace automatically determines whether the floating object is on
% % % % the same page or further away. If the float is on the next page and the next page is to the right of
% % % % the current page, \LaTeX\xspace will insert an appropriate text like \emph{on the facing page}. If we are
% % % % on a right page, \LaTeX\xspace will insert something like \emph{on the next page} or \emph{on the facing page}. If the float is further
% % % % away, a page number will be provided.
% % % % \ispackagee{varioref}
% % % % 
% % % % %Please use \verb+\vref+ for the first reference to a float only.
% % % % 
% % % % 
% % % % \section{\texttt{german} for hyphenation}
% % % % 
% % % % If\is{hyphenation|(}\ispackageb{german} you write things like \head-driven or very long paths like
% % % % {\scshape snysem$|$""loc$|$""cat$|$""head$|$""mod$|$""loc}, \LaTeX\xspace{} does not do hyphenation
% % % % (in the part following the dash).
% % % % 
% % % % german.sty+ provides additional markup that allows for proper hyphenation:
% % % % \begin{verbatim}
% % % % head"=driven
% % % % 
% % % % {\scshape snysem$|$""loc$|$""cat$|$""head$|$""mod$|$""loc}
% % % % \end{verbatim}
% % % % With this markup even long paths like {\scshape snysem$|$loc$|$cat$|$""head$|$""mod$|$""loc$|$""cat$|$""head}
% % % % are typeset properly. Alternatively you my write
% % % % \begin{verbatim}
% % % % {\scshape snysem$|$\-loc$|$\-cat$|$\-head$|$\-mod}
% % % % \end{verbatim}
% % % % which introduces a dash at the place of the linebreak:
% % % % {\scshape snysem$|$\-loc$|$\-cat$|$\-head$|$\-mod$|$\-loc$|$\-cat$|$\-head}.
% % % % 
% % % % If you use german.sty for a book whose primary language is not German, do not forget to
% % % % specify the language you are using. For example, if your book is in US English you have to specify
% % % % the following:
% % % % \begin{verbatim}
% % % % \selectlanguage{USenglish}
% % % % \end{verbatim}
% % % % Otherwise the section name for references comes out in German.
% % % % \is{hyphenation|)}\ispackagee{german}
% % % % 
% % % % \section{Resizing large objects}
% % % % 
% % % % Trees and AVMs often are too big to fit onto one page. The \texttt{langsci} comes with commands for
% % % % shrinking large objects. You may pass your complex object as an argument to \texttt{\oneline} and
% % % % this will scale the object to \linewidth (the remaining space on the current line). There is
% % % % a more clever version of this command: \centerfit. This command checks whether there is
% % % % enough space for an object and if this is the case it centers it in the line. If the object is
% % % % larger than the \linewidth, it is resized to fit the line. This is very handy for typesetting
% % % % figures. You may copy and paste figures to other documents with a different text width without any
% % % % adaptations.
% % % % 
% % % % 
% % % % %% \begin{figure}[htb]
% % % % %% \centerfit{%
% % % % %% \begin{tikzpicture}
% % % % %% \tikzset{level 1+/.style={level distance=3\baselineskip}}
% % % % %% \tikzset{level 2+/.style={level distance=5\baselineskip}}
% % % % %% \tikzset{level 3+/.style={level distance=6\baselineskip}}
% % % % %% \tikzset{level 4/.style={level distance=7\baselineskip}}
% % % % %% \tikzset{level 5+/.style={level distance=5\baselineskip}}
% % % % %% \tikzset{frontier/.style={distance from root=26\baselineskip}}
% % % % %% %% \Tree[.{\ms[np-passive-cx]{ vform & passive \\
% % % % %% %%                             subj & \sliste{ NP\ind{1} }\\[2mm]
% % % % %% %%                             comps & \sliste{ (PP[\type{by}]\ind{2}) }\\
% % % % %% %%                           } }
% % % % %% %%         \ms{ vform & psp \\
% % % % %% %%              subj & \sliste{ NP\ind{2} }\\[2mm]
% % % % %% %%              comps & \sliste{ NP\ind{1} } 
% % % % %% %%            } ]
% % % % %% \Tree[.S
% % % % %%        [.{\ibox{1} NP\ind{2}} \edge[roof]; {the boy} ]
% % % % %%        [.VP\feattab{
% % % % %%                  \subj  \sliste{ \ibox{1} NP\ind{2} },\\
% % % % %%                  \comps  \sliste{  }}
% % % % %%          [.V\feattab{
% % % % %%                  \subj  \sliste{ \ibox{1} NP\ind{2} },\\
% % % % %%                  \comps  \sliste{ \ibox{3} }} was ]
% % % % %%          [.{\ibox{3} VP\feattab{
% % % % %%                  \vform \type{passive},\\
% % % % %%                  \subj  \sliste{ \ibox{1} NP\ind{2} },\\
% % % % %%                  \comps  \sliste{ }}} \edge node[auto=left]{Passive Construction};
% % % % %%            [.V\feattab{
% % % % %%                  \vform \type{psp},\\
% % % % %%                  \subj  \sliste{ NP },\\
% % % % %%                  \comps  \sliste{ NP\ind{2} }} 
% % % % %%              [.V\feattab{
% % % % %%                  \vform \type{psp},\\
% % % % %%                  \subj  \sliste{ NP },\\
% % % % %%                  \comps  \sliste{ NP\ind{2}, \ibox{4} }} given ] \edge node[auto=left]{Schema for Passive Participles};
% % % % %%              [.{\ibox{4} NP} \edge[roof]; { the ball } ] ] ] ] ] 
% % % % %% \end{tikzpicture}
% % % % %% }
% % % % %% \caption{\label{fig-the-boy-was-given-the-ball-tseng}Analysis of \emph{The boy was given the ball} according to \citet{Tseng2007a}}
% % % % %% \end{figure}
% % % 
% % 
% % \section{Intonation}
% We have created a small command for intonation pattern shown over examples. Use \verb+\intline{height}{text}+ to add a line of the specified height directly over the text. Use \verb+\dline{height}{slope}{length}+ to add a descending line. Finding the right slope and length requires some testing. 
% 
% \newpage 
% \begin{verbatim} 
% \ea  
% \gll
% \intline{18}{iˈ}%
% \intline{20}{we}%
% \intline{16}{ra} %
% \intline{14}{mu}%
% -\intline{14}{ˈep} %
% \intline{10}{maa} % 
% \intline{12}{ˈuu}%
% \dline{12}{3}{24}p-i-nen \\
% coconut  scrape-\textsc{ss.seq} food  cook-\textsc{Np-fu}.1s      \\
% \glt`I will scrape a coconut and cook food.'
% \z 
% \end{verbatim}
% 
% \ea  
% \gll
% \intline{18}{iˈ}%
% \intline{20}{we}%
% \intline{16}{ra} %
% \intline{14}{mu}%
% -\intline{14}{ˈep} %
% \intline{10}{maa} % 
% \intline{12}{ˈuu}%
% \dline{12}{3}{24}p-i-nen \\
% coconut  scrape-\textsc{ss.seq} food  cook-\textsc{Np-fu}.1s      \\
% \glt`I will scrape a coconut and cook food.'
% \z 
