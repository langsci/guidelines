\chapter{Indexing}
\lsp books have an obligatory Name Index and an obligatory Subject Index. The Language Index is optional and should be used if your work makes reference to more than one language. 
For the various ways to add entries to the index, refer to \tabref{tab:latex:indexentriese}. For every index, there are two commands. The shorter one adds a term to the relevant index but does not change your text. This is useful if the term you want to add to your index does not appear in exactly the same way in the text. If the term is indeed identical, you can use the command with an extra \verb+i+.

\begin{table}[h]
\caption{Commands for creating index entries.}
\label{tab:latex:indexentriese}
 \begin{tabular}{p{1.9cm}>{\small\raggedright}p{6.5cm}l}
  \lsptoprule
  type & \upshape\normalsize command & indexed term \\
  \midrule
  Subject Index& \verb+Nominalized sentences+ \verb+\is{nominalization} are+ \verb+common.+ & nomina\-lization \\
  \mbox{Subject Index} identical& \verb+... while \isi{nominalization}+ \verb+ is less frequent ...+  & nomina\-lization \\\\
  Language Index & \verb+Varieties of Chinese+ \verb+\il{Sinitic languages}+ \verb+differ in that ...+ & Sinitic languages \\
  Language Index identical& \verb+The \ili{Sinitic languages},+ \verb+however, ... + & Sinitic languages \\\\
  Author Index & \verb+In Homeric \ia{Homer}+ \verb+language, ... + & Homer\\
  \mbox{Author Index} identical & \verb+This contradicts+ \verb+\iai{Homer}, who had+ \verb+advocated ...+ & Homer \\
  \lspbottomrule
 \end{tabular}
\end{table}

If there are two or more entries on subsequent pages, the index generation will automatically produce a range. So, instead of `33,34,35,36', it will print out `33--36'. You can produce ranges yourself by using \verb+\is{someterm|(}+ for the start and  \verb+\is{someterm|)}+ for the end of the range. 

Do not use the indexing commands directly before punctuation as it can produce unwanted white space. Put it after the punctuation instead.

% If you compile your document with the option \verb+draftmode+ all indexed terms will show up in the margins.

When your are done with adding index terms to your document, the following commands will produce the Subject Index and the Language Index
\begin{verbatim}
  makeindex -o main.lnd main.ldx
  makeindex -o main.snd main.sdx
\end{verbatim}

In order to create the author index, run

\begin{verbatim}
  sed -i s/.*\\emph.*// main.adx  
  makeindex -o main.and main.adx
\end{verbatim}

 
% The author index should be produced automatically.
% \begin{verbatim} 
% authorindex -i -p yourfilename.aux > yourfilename.bib.adx
% sed 's/|hyperpage//' yourfilename.adx > yourfilename.txt.adx 
% cat yourfilename.bib.adx yourfilename.txt.adx > yourfilename.combined.adx
% makeindex -o yourfilename.and yourfilename.combined.adx
% \end{verbatim}

After the creation of the indexes, check for every index whether it contains only terms that should be found in this index (no languages in Subject Index and vice versa). Furthermore, check that every concept has exactly one entry in the index. It is easy to index the same concept once in the singular and then again in the plural, or once with a hyphen and once without. 

For the Name Index, make sure that every author has exactly one entry. Common errors include abbreviated names, middle initials which are present in one entry but absent in another, different transcriptions of a name, and diacritics. These issues are fixed by opening your bibliography file and conforming the names of the authors there.

After your indexed terms are final, check the Name Index for terms which are not names. This happens if one of your cited works has an institution as the author. Open the \verb+.adx+ file and remove that entry. Be aware that a recompilation of your index will overwrite your changes. You can also adapt the \verb+Makefile+ to automatically remove such items in future compilations. 

\subsection{See also}
We provide a file \verb+localseealso.tex+ where you can list crossreferences in the index. 

% Check your index for overlong lines. Use hyphenations \verb+\mbox{...}+ or \verb+\newline+s in the \verb+.adx+ file to repair these. Again, a recompilation of the index will overwrite your changes.
% Ja und ein Index ist auch sowas wie ein Inhaltsverzeichnis. Man kann dann zum Beispiel auf einen Blick sehen, dass es in dem Buch viel um Extraposition geht. Das ist mit dem Autorenindex �brigens genauso. Den braucht man theoretisch noch weniger als den Subject-Index, aber man sieht dann eben auf einen Blick, dass Haider viel zitiert wurde. 

\laundrybox{Indexes}{
\laundryitem Subject index is there
\laundryitem Language index is there
\laundryitem Author index is there
\laundryitem No duplicate terms in subject index
% \laundryitem Subject index has no overlong lines
\laundryitem No duplicate terms in language index
% \laundryitem Language index has no overlong lines
\laundryitem Author index has no titles in it
\laundryitem Author index has no institutions in it
\laundryitem No duplicate authors in author index
% \laundryitem Author index has no overlong lines
}