\chapter{New series}
\section{Introduction}
Language Science Press is keen to evaluate proposals for new book series in all areas
of linguistics and neighboring fields. The following notes aim at
helping you prepare a proposal that can be fully and rapidly assessed
by the managing board. We kindly ask you to include the following
aspects into your proposal, giving as specific information as possible.
Your proposal should give the managing board a clear idea of the aims
and scope of the proposed series. Proposal may vary in length. We
expect proposals ranging between five to eight pages. Your initial
submission should include a letter of introduction.


\section{The proposed series in a nutshell}

Please give the proposed title of the new series and describe the
aims and scope of the proposed series briefly in two or three sentences.


\section{Content and contribution of the series}

Please give an explicit statement on the aims and scope of the book
series, addressing also the following aspects.

What sub-discipline or sub-disciplines of linguistics will be covered?
Which kind of topics would you like so see addressed in potential
volumes?

Please explain how the proposed book series will fill a gap in existing
publication programs for the scholarly community addressed by the
series. Have there been new developments in the fields which have
caused the need for a new series?

Please provide a brief list of keywords relating to the topics to
be covered in potential volumes. It might also be worthwhile to comment
on your choice of a title for the proposed series.


\section{Readership}

What is the main audience of the new book series? Does the series
aim at academic specialists, at graduate or undergraduate students
or at specialists in areas other than that of the editors or authors?


\section{Competition}

Please explain how the series differs from existing series in the
area. What is its unique focus?

What is the existing competition? Please give details of the most
relevant existing series in the area, providing publisher and editors.
How does the proposed series relate to existing programs? How does
it stand out from the competition?


\section{Form of publications to appear in the proposed series}

Which kind of publications will be included in the proposed series?
Do you expect to publish monographs, edited books, textbooks, theses,
handbooks and/or reference manuals?

How long will books be? Please give an approximate word count.

How many volumes do you expect to publish per year? Do you anticipate
a limitation of the total number of books appearing in the series?

Will the proposed series accept only publications written in English,
or do you expect to publish contributions written in other languages?
If so, which languages other than English do you consider suitable
for publications appearing in the proposed series? How do you plan
to deal with possible difficulties arising with respect to publications
in English authored by non-native speakers?


\section{Editorial policies}

Please give details on how you propose to administer the planned series.
What role do you expect to play as editors, and what part do you plan
to assign to the editorial board? Whom would you suggest as members
of the editorial board?

As series editors you are responsible for ensuring the quality of
each volume appearing in the series. In particular, it will be crucial
that each proposed volume be commented on by at least two referees
besides the series editor or editors. Please explain how you propose
to accomplish this task. How do you propose to make editorial decisions?

Please specify how you project to approach potential authors or editors
for future volumes in the proposed series. Do you plan to commission
contributions on particular topics?

What lead time do you anticipate for each volume? How do you propose
to ensure that editorial decisions and preparation of manuscripts
will be accomplished in a timely manner?


\section{About the editors}

Please list for each editor present academic interests, position and
professional affiliation, including a list of recent publications
which are relevant to the area of the proposed series. If applicable,
specifically mention previous editorial experience. We ask you to
make sure to include the following details for each prospective editor:
Name, mailing address, work phone, email. Please try to explain as
specifically as possible in which respect you are the right editor
or editors for the proposed series.


\section{Proposed first volume or volumes of the series}

Please give details on the proposed first volume or volumes of the
planned series, including the following information: Provisional title,
volume editor or editors (name plus affiliation), brief synopsis of
the volume, provisional table of contents, and proposed schedule for
completion of the typescript. It would be worthwhile to elaborate
on how the planned first volume or volumes are intended to set the
pitch for further contributions in the proposed series.

\section{Languages}

Please indicate the languages in which you will accept submissions and provide details about the
expected readership, if you want to publish in languages other than English.

Language Scienece Press accepts submissions in English, German, French, Spanish,
and Portugese, so series can choose to accept a subset of these languages.

%% \section{Potential referees for this proposal}

%% Please give names and addresses of two to five scholars in the area
%% of the proposed series who can be contacted to comment on the proposal.
%% Please specify whether they are aware of the proposed series.

\section{Financial support}

Currently we have nine series, for the enterprise to scale up, it is important that series editors
take some responsibility and help as much as they can. De Gruyter requests camera ready copies or
charges 1.500€ for submissions to certain series. We do not charge any money but it is not possible
to handle everything centrally.

Series are expected to deliver \latex manuscripts. We help series editors to set up the conversion
workflow, but certain commitment and resources at the site of the series editors is required. Please
state that you can provide manuscripts typeset in \latex and what other resources you can provide.


\section{What we do not publish}

We do not publish scholarly work in fields other than linguistics
and neighboring fields. We do not publish books for the general reader,
\textit{festschriften}, unrevised dissertations, autobiography or
fiction.\\


To discuss your series proposal, please contact sebastian.nordhoff@langsci-press.org.  He will help you with any questions you might have. The final proposal has to be sent to the press directors.



\section{Acceptance}
Upon acceptance of your proposal, you have to choose a color for your series, an acronym, and a short description of \emph{Aims and Scope}.
All books have a series overview page. This usually lists the editors. You can also add additional material, like a logo or a short description. About half the page should be reserved for a list of titles that have appeared in the series. Have a look at existing series for inspiration.

