\chapter{Edited volumes}
\section{Workflow}

Edited volumes are submitted as a whole. It is thus the task of the volume editor to assure the integration of the various chapters. It is highly recommended that all authors use the templates provided (Word, LibreOffice, \LaTeX). 
Chapter authors should use the template for papers. The editor should download the skeleton for edited volumes and add all author's files to the folder \verb+chapters+ when they are ready. See our screencast on edited volumes available at \url{http://langsci-press.org/forAuthors}. In the skeleton, the files should be included via \verb+\includepaper{chapters/kim.tex}+. Make sure that the options \verb+collection+ and \verb+collectionchapter+ are used in the preamble of your master file. If you use our skeleton for edited volumes, this is already done for you. 
The chapter templates for edited volumes contain fields for epigrams and abstracts. While abstracts should be used, epigrams should rather be avoided as they clutter the page in combination with the abstract. 

All chapters will have their own list of references, but all lists will be built using the same {\bibtex} file. This is done in order to avoid that two authors cite the same work differently. 
% In order to compile the bibliographies for the individual chapters, you have to run {\BibTeX} on the relevant \verb+blx.aux+ file which will show up after compiling the master file. There is a Makefile in the skeleton which includes all relevant commands, and you may also use the script \verb+bibtexvolume.sh+ shipped with the skeleton.
     
\section{Special style rules for edited volumes}
Some special rules apply to the chapter of edited volumes:
\begin{itemize}
\item Each paper should start with a short abstract
\item A paper may have a special unnumbered section Acknowledgements just after the last numbered section. This is preferable to putting the acknowledgements into the footnotes.
\item A paper may have a special unnumbered section Abbreviations (or similar) just before the References. This is strongly preferred to listing the abbreviations in a footnote.
\item Each paper has its own list of references (unnumbered section labeled References).
\item Chapter numbers should not be used in numbering tables and figures within such chapters.
\end{itemize}