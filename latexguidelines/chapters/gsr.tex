\chapter{Generic Style Rules}
Toward discipline-specific text-structure style rules
Scientists have certain style rules for structural aspects of their research papers and
monographs, which in the past were primarily set and enforced by the academic publishers.
But in the 21st century, science is increasingly international and research papers spread
easily even without the publishers’ copy-editors and style watchers.
This does not mean that there is no need for style rules anymore. It makes our research
and our publication activities easier if we agree on a common set of conventions for
frequently recurring structural aspects of our writings, of the sort that are commonly
prescribed in journal style sheets (called TEXT-STRUCTURE style here). But it is inefficient if
these rules are set by individual journal editors or publishers, because scientists usually
publish in diverse venues, and being forced to apply different style rules in different papers
is an unnecessary burden on the authors. If linguists could agree on a set of rules, then
linguistics publishers would probably be happy to adopt them sooner or later, because they
would no longer have to worry about enforcing their house styles.
For the specific case of formatting rules for bibliographical references, this has already
happened: In 2007, a number of linguistics journal editors agreed on a “Unified Style Sheet
1for Linguistics”, and these rules for bibliographical reference style have been widely
adopted, not just for journals, but also for linguistics books.
Another aspect of form style has been widely adopted: The Leipzig Glossing Rules for
2interlinear morpheme-by-morpheme glosses. Quite a few journals and publishers now
recommend or prescribe their use, and many authors refer to them. The Leipzig Glossing
Rules are now typically taught in linguistics classes, and more and more linguists find it
normal that knowing them is part of their disciplinary competence.
The following style rules for formal aspects of linguistics papers were formulated in the
same spirit. Linguistics papers have been converging in their text-structure style over the
last 20 years anyway, so while there are still a number of things that are sometimes done
differently, none of the following rules will be particularly controversial. In most cases, the
rules reflect majority usage, and none of the rules represents an innovation. Where they do
not appear to reflect majority usage, they are always simpler than the majority usage (e.g.
eliminating poorly motivated exceptions to general rules). Text-structure style should
primarily be practical and can often leave aside purely aesthetic considerations.
The present style rules focus on special conventions for linguistics-specific aspects like
numbered example sentences and the representation of material from other languages, but
also provide guidance for many other aspects of text-structure style which should be
uniform across a paper, or an edited volume, and probably also across the papers of a
journal. The rules do not say anything about more specific notational conventions that are
relevant only to certain subcommunities of linguists, e.g. for syntactic tree representation,
	
 	
 	
 	
 	
 	
 	
 	
 	
 	
 	
 	
 	
 	
 	
 	
 	
 	
 	
 	
 	
 	
 	
 	
 	
 	
 	
 	
 	
 	
 	
 	
 	
 	
 	
 	
 	
 	
 	
 	
 	
 	
 	
 	
 	
 	
 	
 	
 	
 	
 	
 	
 	
 	
 	
	
 
1
https://linguistlist.org/pubs/journals/	
  
2
 http://www.eva.mpg.de/lingua/resources/glossing-rules.php
	
 
 2	
  
transcription of spoken dialogue, optimality-theoretic tableaux, and so on. More specific
documents would be needed for each of these.
Nothing is said here about typographic features such as font type, font size, indentation
and line spacing, let alone about margin and paper size. Journal style sheets often specify
these as well, but they seem increasingly irrelevant. There is also nothing about English
spelling, generic pronoun use or date format, as these are issues that are not specific to
linguistics and rarely present problems in editing linguistics papers. The present rules are
also different from journal style sheets in that they do not give instructions for submitting a
paper for typesetting, but concern the form of a paper as it should look to the reader. The
reason for this is that while submission rules will continue to depend on diverse typesetting
technologies, there is no reason why linguists should not agree on the way certain formal
aspects of their papers should appear to the reader.
Occasionally the rules below make reference to some other prominent stylesheets,
especially those of the journals Language (LSA), Linguistic Inquiry (MIT Press), Journal of
Linguistics (Cambridge), and the ”Stylesheet for De Gruyter Mouton journals”.3 There is no
systematic comparison, but some cross-references seem useful to make readers aware of
certain salient differences between styles.
1. Parts of the text
The text of an article begins with the title, followed by the name of the author and the
affiliation. Articles are preceded by an abstract of 100–300 words. About five keywords are
given.
Articles are subdivided into numbered sections (and possibly subsections), each of which
has a heading. The numbering always begins with 1 (Section 1: 1.1, 1.2, Section 2: 2.1, 2.2,
etc.), so that 0 never occurs in section numbering.
More than three levels of subsections should only be used in special circumstances. If
this cannot be avoided, unnumbered subsection headings are possible.
The last numbered section may be followed by several optional sections (Sources,
Acknowledgements, Abbreviations), and by one or more sections called Appendix (A, B,
etc.).
The last part is the alphabetic list of bibliographical references (References). For the
style of references, see §12 below.
If a (sub-)section has (sub-)subsections, there must be minimally two of them, and they
must be exhaustive. This means that all text in a chapter must belong to some section, all
text within a section must belong to some subsection, and so on. A short introductory
paragraph is allowed by way of exception.
Section headings do not end with a period, and have no special capitalization (see §2).
For the parts of monographs and edited volumes, see §17 below.
	
 	
 	
 	
 	
 	
 	
 	
 	
 	
 	
 	
 	
 	
 	
 	
 	
 	
 	
 	
 	
 	
 	
 	
 	
 	
 	
 	
 	
 	
 	
 	
 	
 	
 	
 	
 	
 	
 	
 	
 	
 	
 	
 	
 	
 	
 	
 	
 	
 	
 	
 	
 	
 	
 	
	
 
3
 Language:
 http://www.linguisticsociety.org/lsa-publications/language
Linguistic Inquiry:
 http://www.mitpressjournals.org/userimages/ContentEditor/1377619488121/
LI\_Style\_Sheet\_8.20.13.pdf
Journal of Linguistics:
 http://assets.cambridge.org/LIN/LIN\_ifc.pdf
De Gruyter Mouton:
 http://www.degruyter.com/staticfiles/pdfs/mouton\_journal\_stylesheet.pdf
	
 
 3	
  
2. Capitalization
Sentences, proper names and titles/headings/captions start with a capital letter, but there is
no special capitalization (“title case”) within English titles/headings, neither in the article
title nor in section headings or figure captions.4 Book titles in the references do not have
special capitalization either (but English journal titles and series titles do, as these are
treated as proper names). Thus, we have:
1.1 Overview of the issues
(NOT: Overview of the Issues)
Figure 3. A schematic representation of the workflow
(NOT: A Schematic Representation of the Workflow)
Anderson, Gregory. 2006. Auxiliary verb constructions. Oxford: Oxford University Press.
(NOT: Auxiliary Verb Constructions)
Capitalization is used only for parts of the article (chapters, figures, tables, appendixes)
when they are numbered, e.g.
as shown in Table 5
more details are given in Chapter 3
this is illustrated in Figure 17
5
Capitalization is also used after the colon in titles, i.e. for the beginning of subtitles:
Clyne, Michael (ed.). 1991. Pluricentric languages: Different norms in different nations. Berlin: Mouton
de Gruyter.
3. Italics
Italics are used for the following purposes:6
• For all object-language forms (letters, words, phrases, sentences) that are cited within
the text or in numbered examples (see §10), unless they are phonetic transcriptions
or phonological representations in IPA.
• For book titles, journal titles, and film titles.
• When a technical term is referred to (in such contexts, English technical terms are
thus treated like object-language forms), e.g.
the term quotative is not appropriate here
I call this construction quotative.
	
 	
 	
 	
 	
 	
 	
 	
 	
 	
 	
 	
 	
 	
 	
 	
 	
 	
 	
 	
 	
 	
 	
 	
 	
 	
 	
 	
 	
 	
 	
 	
 	
 	
 	
 	
 	
 	
 	
 	
 	
 	
 	
 	
 	
 	
 	
 	
 	
 	
 	
 	
 	
 	
 	
	
 
4
Note that the title of the present document has special capitalization because the Leipzig Style Rules
for Linguistics is a name.
5	
 Note that the character § is used instead of Section, see §13 below.
6	
 Italics are not used for commonly used loanwords such as ad hoc, façon de parler, e.g., et al., Sprachbund.
	
 
	
 
 4	
  
• For emphasis of a particular word that is not a technical term, e.g.
This is possible here, but only here.
• For emphasis within a quotation, with the indication [emphasis mine].
4. Small caps
Small caps are used to draw attention to an important term at its first use or definition,7
e.g.
On this basis, the two main alignment types, namely NOMINATIVE-ACCUSATIVE and ERGATIVE-
ABSOLUTIVE, are distinguished.
Small caps are also used for category abbreviations in interlinear glossing (see §8, §10), and
they may be used to indicate stress or focusing in example sentences:
(1) John called Mary a Republican and then SHE insulted HIM.
5. Boldface and other highlighting
Boldface can be used to draw the reader’s attention to particular aspects of a linguistic
example, whether given within the text or as a numbered example. An example is the
relative pronoun dem in (4) in §10 below.
Full caps and underlining are not normally used for highlighting. Exceptionally,
underlining may be used to highlight a single letter in an example word, and in some other
cases where other kinds of highlighting do not work.
6. Quotation marks
Double quotation marks are used for distancing, in particular in the following situations:
• When a passage from another work is cited in the text, e.g.8
According to Takahashi (2009: 33), “quotatives were never used in subordinate clauses in Old
Japanese”.
• When a technical term or other expression is mentioned that the author does not
want to adopt, e.g.9
This is sometimes called “pseudo-conservatism”, but I will not use this term here, as it could lead to
confusion.
Single quotation marks are used exclusively for linguistic meanings,10 e.g.
	
 	
 	
 	
 	
 	
 	
 	
 	
 	
 	
 	
 	
 	
 	
 	
 	
 	
 	
 	
 	
 	
 	
 	
 	
 	
 	
 	
 	
 	
 	
 	
 	
 	
 	
 	
 	
 	
 	
 	
 	
 	
 	
 	
 	
 	
 	
 	
 	
 	
 	
 	
 	
 	
 	
	
 
7
This is in line with the Language stylesheet. The De Gruyter stylesheet requires italics for this purpose.
8
 But note that block quotations do not have quotation marks.	
 
9	
 Alternatively, italics could be used here, cf. §3.
	
 
 5	
  
Latin habere ‘have’ is not cognate with Old English hafian ‘have’.
Quotes within quotes are not treated in a special way.
Note that quotations from other languages shouldin a footnote if they are longer).
betranslated (inline if they are short,
7. Other punctuation matters
The n-dash (–) surrounded by spaces is used for parenthetical remarks – as in this example
– rather than the m-dash (—). The n-dash is also used for number ranges, but not
surrounded by spaces (e.g. 1995–1997).
Ellipsis in a quotation is indicated by [...].
Angle brackets are used for specific reference to written symbols, e.g. the letter <q>.
8. Abbreviations
Abbreviations of uncommon expressions should be avoided in the text. Language names
should not normally be abbreviated.
The use of abbreviations is desirable for grammatical category labels in interlinear
morpheme-by-morpheme translations. (The Leipzig Glossing Rules include a standard list
of frequently used and widely understood category label abbreviations.)
When a complex term that is not widely known is referred to frequently, it may be
abbreviated (e.g. DOC for “double-object construction”). The abbreviation should be given
both in the text when it is first used and at the end of the article in the Abbreviations
section.
Abbreviations of uncommon expressions are not used in headings or captions, and they
should be avoided at the beginning of a chapter or section.
9. In-text citations
Published works can be cited by including the author-year name of the work as an element
in the primary text (as in the first example below), or by backgrounding it in parentheses
(as in the second example below).
Thomason \& Kaufman (1988: 276-280) point out that the northern dialects of English show
more morphological innovations (and are morphologically more simple) than the southern
English dialects.
The notation we use to represent this is borrowed from theories according to which φ-features
occur in a so-called feature geometry (Gazdar \& Pullum 1982).
The full bibliographical references corresponding to all citations are listed alphabetically at
the end of the work. The author-year name consists of the author’s surname and the
	
 	
 	
 	
 	
 	
 	
 	
 	
 	
 	
 	
 	
 	
 	
 	
 	
 	
 	
 	
 	
 	
 	
 	
 	
 	
 	
 	
 	
 	
 	
 	
 	
 	
 	
 	
 	
 	
 	
 	
 	
 	
 	
 	
 	
 	
 	
 	
 	
 	
 	
 	
 	
 	
 	
 	
 	
 	
 	
 	
 	
 	
 	
 	
 	
 	
 	
 	
 	
 	
 	
 	
 	
 	
 	
 	
 	
 	
 	
 	
 	
 	
 	
 	
 	
 	
 	
 	
 	
 	
 	
 	
 	
 	
 	
 	
 	
 	
 	
 	
 	
 	
 	
 	
 	
 	
 	
 	
 	
 	
 	
 	
 	
 	
 	
 	
 	
 	
 	
 	
 	
 	
 	
 	
 	
 	
 	
 	
 	
 	
 	
 	
 	
 	
 	
 	
 	
 	
 	
 	
 	
 	
 	
 	
 	
 	
 	
 	
 	
 	
 	
 	
 	
 	
 	
 	
	
  
10
 The distinction between single and double quotation marks is not made by Language and Journal of
Linguistics, but is very useful and is practiced widely (e.g. required by the Linguistic Inquiry and De Gruyter
Mouton stylesheets).
	
 
 6	
  
publication year (with no comma between them),11 followed by page numbers. The page
numbers may only be omitted if the citation concerns the entire work. In primary citations,
the year (plus page numbers) is enclosed in parentheses, while in backgrounded citations,
12
the year is not in parentheses.The page numbers follow the year after a colon and a space and are given with complete
numbers (no digits dropped). The use of “f” or “ff” (for ‘and following’) is not acceptable.
When there are two authors, the ampersand \& (rather than and) is used, and when
there are more than two authors, the most normal author-year name includes only the first
surname plus et al. (though the full list of authors may be given if this helps the reader).13
Sperber \& Wilson (1986)
Bannard et al. (2009)
= Bannard, Lieven \& Tomasello (2009)
When multiple citations are listed in parentheses, they are separated by semicolons,14 and
they are normally listed in chronological order.
Speakers rely heavily on formulaic chunks or “prefabs” during speech comprehension and production
(Pawley \& Syder 1983; Sinclair 1991; Erman \& Warren 2000; Bybee 2006; see Wray 2002 for a
broader historical review).
When multiple works by the same authorand the years are separated by semicolons.
are cited, the author name need not be repeated,
While Hawkins (2004; 2014) has argued for a Minimize Domain principle of language performance,
other authors have tried to explain the observed effects in purely grammatical terms.
Previous empirical studies report that object fronting in these languages occurs under the same
contextual conditions for canonical transitive verbs and experiencer-object verbs (see Verhoeven 2008b;
2010a for Turkish and Chinese).
Instead of page numbers, chapter numbers or section numbers may be given (e.g. Auer
2007: Chapter 7, Matras 2009: §6.2.2).
10. Numbered examples
A hallmark of many linguistics articles is the use of numbered examples. Unless they are
from English (or more generally, the language of the article), they must be glossed and
translated. Glossing refers to the use of interlinear word-by-word or morpheme-by-
morpheme translations, as described in detail in the Leipzig Glossing Rules.
	
 	
 	
 	
 	
 	
 	
 	
 	
 	
 	
 	
 	
 	
 	
 	
 	
 	
 	
 	
 	
 	
 	
 	
 	
 	
 	
 	
 	
 	
 	
 	
 	
 	
 	
 	
 	
 	
 	
 	
 	
 	
 	
 	
 	
 	
 	
 	
 	
 	
 	
 	
 	
 	
 	
	
 
11
Using a comma between author and year is widespread in other disciplines, but in linguistics it seems to be
mostly confined to the Elsevier journals.
12
In line with majority usage, no special distinction is made here between the author and the published work
(contrasting with the Language and Linguistic Inquiry style sheets).
13
Language now uses and colleagues rather than et al. (when the author rather the the work is referred to), but
the latter is extremely widespread across the disciplines. (It derives from Latin et alii ‘and others’.)
14
Linguistics publications more frequently use a comma in such listings, but the semicolon is much more
frequent in other disciplines, so it is adopted here.
	
 
 7	
  
Example numbers are enclosed in parentheses. When there are multiple examples (“sub-
examples”) under a single number, they are distinguished by the letters a, b, etc. The text
of numbered examples is in italics, just like the text of in-line examples (§3).
(2)
 a. She saw him.
b. He saw her.
15Cross-references to examples use parentheses as well, but immediately inside
parentheses these can be omitted:
As shown in (6) and (8-11), this generalization extends to transitive constructions, but (29b) below
constitutes an exception.
In all other environments, the stress is on the second syllable (see 15a-d).
When an example is from a language other than the language of the main text, it is
provided with an interlinear gloss (with word-by-word alignment) in the second line, as
well as an idiomatic translation in the third line, e.g.
(3) Icelandic
Storm-ur-inn
 rak
 bát-inn
 á land.
storm-NOM - DEF drove boat.ACC-DEF on land
‘The storm drove the boat ashore.’
The precise conventions for interlinear glossing are given in the Leipzig Glossing Rules,
which have become a worldwide standard. The most important principle is that each
element of the primary text corresponds to an element in the gloss line, and boundary
symbols (especially the word-internal boundary symbol - and the clitic boundary symbol =)
have to be present both in the primary text and in the gloss. Abbreviated category labels are
set in small capitals, and the idiomatic translation is surrounded by single quotes. A list of
abbreviations is provided at the end of the article (or at the beginning of a monograph).
Example sentences usually have normal capitalization at the beginning and normal
punctuation (normally a period) at the end. The gloss line has no capitalization and no
punctuation. The idiomatic translation again has normal capitalization and punctuation, as
seen in (3) above. When the example is not a complete sentence, as in (4), there is no
capitalization and no punctuation.
(4) das Kind,
 dem
 du
 geholfen hast
the child.NOM who.DAT you.NOM helped have
‘the child that you helped’
When the language is not normally used as a written language, the primary text may lack
initial capitalization and normal punctuation, e.g.
	
 	
 	
 	
 	
 	
 	
 	
 	
 	
 	
 	
 	
 	
 	
 	
 	
 	
 	
 	
 	
 	
 	
 	
 	
 	
 	
 	
 	
 	
 	
 	
 	
 	
 	
 	
 	
 	
 	
 	
 	
 	
 	
 	
 	
 	
 	
 	
 	
 	
 	
 	
 	
 	
 	
	
 
15
This is the most widespread practice, although the Language stylesheet omits the parentheses in cross-
references.
	
 
 8	
  
(5) Hatam
a-yai
 bi-dani mem di-ngat i
2SG-get to-me for
 1SG-see Q
‘Would you give it to me so that I can see it?’ (Reesink 1999: 69)
When multiple languages are mentioned in a single text, the name of the language may be
given in the line next to the example number, as in (5) and (6a-b).
(6) Sakha
a. En bytaan buol-uoq-uŋ
you slow
 be-FUT-2SG
‘You will be slow.’ (Baker 2012: 7)
b. *En bytaan-yaq-yŋ
you slow-FUT-2SG
(‘You will be slow.’) (Baker 2012: 7)
Ungrammatical examples can be given a parenthesized idiomatic translation, as in (6b). A
literal translation may be given in parentheses after the idiomatic translation, e.g.
(7) Japanese
Tsukue no ue ni hon
 ga aru.
table
 GEN top at book SUBJ be
‘There is a book on the table.’ (Lit. ‘At the top of the table is a book.’)
The object-language text may be given in two lines, one unanalyzed (“surface“) line, and an
analyzed line (in roman type), which may contain a more abstract representation, e.g.
(8) Karbi
amatlo la kroikrelo
amāt=lo
 là
 krōi-Cē-lò
and.then=FOC this agree-NEG-RL
‘And then, she disagreed.’ (Konnerth 2014: 286)
When a numbered example is not glossed and translated (i.e. in English works, when it
is from English), it may be in roman (non-italic) type. Thus, (2a-b) above could
alternatively be printed in roman.
Angle brackets are never set in italics, even when the text is in italics.
11. Source indications
Sources of linguistic examples are standardly given directly after the idiomatic translation,
as in the following examples (see also (5-6) and (8) above):
(9) Luganda
Maama a-wa-dde
 taata
 ssente.
Mother she.PRS-give-PRF father money
‘Mother has given father money.’ (Ssekiryango 2006: 67)
	
 
 9	
  
(10) Jalonke
I
 sig-aa
 xon-ee
 ma.
2SG go-IPFV stranger-DEF at
‘You are going to the stranger.’ [Mburee 097]
If the source is not a bibliographical reference, but is the name of a text (perhaps
unpublished), as in (10), the source is given in square brackets and the article must contain
a special section at the end where more information about the sources is given.
12. Tables and figures
Tables and figures are numbered consecutively (Table 1, Table 2; Figure 1, Figure 2, etc.).
They must be mentioned in the running text and identified by their numbers. They appear
in the text as close as possible to the place where they are mentioned.
Each table and each figure has a caption that ideally is not longer than a line. The
caption precedes a table and follows a figure. It is not followed by a period, and does not
have special capitalization, like section headings.
Tables generally have a top line and a bottom line plus a line below the column headers,
e.g.
Table 3: English (British National Corpus of English)
SG
 PL
 \% OF SG
person
 24671 persons 4034
 86\%
house
 49295 houses 9840
 83\%
hare
 488
 hares
 136
 78\%
bear
 1182
 bears
 611
 65\%
feather
 487
 feathers 810
 38\%
Further explanation may go below the table, or below the caption of a figure.
Footnotes within a table use the footnote reference characters a, b, c and are given
immediately below the table (not at the bottom of the page).
13. Cross-references in the text
Cross-references to chapters, tables, figures or footnotes use the capitalized names for these
items (e.g. Chapter 4, Figure 3, Table 2, Footnote 17). Abbreviations like “Fig. 3”, “Ch. 4”,
or “n. 17” are not used.
Cross-references to sections use the § character (e.g. §2.3).
14. Footnotes
The footnote reference number normally follows a period or a comma, though
exceptionally it may follow an individual word.
Footnote numbers start with 1. The acknowledgements are printed as a separate section
following the body of the text, not as a footnote. Likewise, abbreviations and other
notational conventions are given in a separate section (following the acknowledgements, see
§1 above).
	
 
 10	
  
Numbered examples in footnotes have the numbers (i), (ii), etc. If there are sub-
examples, they have the numbers (i.a), (i.b), etc.
15. Non-Latin scripts
All forms in languages that are not normally written with the Latin alphabet (such as
Japanese or Armenian) should (also) be given in transcription or transliteration.
When the article is entirely about a particular language, the original script should not be
omitted, at least in numbered examples.
Non-Latin forms need not be printed in italics.
16. List of references
The list of referenceslisted alphabetically.
16.1. General points
at the end of an article hasthe heading References. The entries are
For the formatting, the Leipzig Style Rules follow the 2007 “Unified Style Sheet for
Linguistics” in almost all respects. Four very minor differences (which simplify the rules by
removing exceptions) are noted in footnotes 18-21 below.16
It should be noted especially that
• full given names of all author and all editors should be included (unless the author
habitually uses abbreviated given names, e.g. R. M. W. Dixon)
• page numbers are obligatory, but issue numbers of journals and series titles are optional
(though recommended)
• journal titles are not abbreviated
• main title and subtitle are separated by a colon, not by a period.
16.2. Standard parts and standard reference types
A reference consists of the standard parts given in Table 1 (some of them are optional):
author list, year, article title, editor list, publication title, volume number, issue number,
series, page numbers, city, publisher. Nonstandard parts may follow in parentheses.
There are four standard reference types: journal article, book, article in edited book,
thesis. Works that do not fit easily into these types should be assimilated to them to the
extent this is possible. Different reference types make use of different parts, as shown in
Table 1.
	
 	
 	
 	
 	
 	
 	
 	
 	
 	
 	
 	
 	
 	
 	
 	
 	
 	
 	
 	
 	
 	
 	
 	
 	
 	
 	
 	
 	
 	
 	
 	
 	
 	
 	
 	
 	
 	
 	
 	
 	
 	
 	
 	
 	
 	
 	
 	
 	
 	
 	
 	
 	
 	
 	
	
 
16
 Perhaps the strongest justification for simple rules is that the references should be automatically
parsable (e.g. by Google Scholar), and correct and complete author names should be extractable. In the
modern age, this is crucial for scientometric and hence career-building purposes.
	
 
 11	
  
Table 1: Standard parts of bibliographical references
author
 year.
 article
 editor
 publica
 volume
 page
 city:
 pub-
list.
 title.
 list.
 -tion
 number
 num
 lisher.
title(.)
 -bers.
journal
 *
 *
 *
 *
 *
 *
article
book
 *
 *
 *
 *
 *
article
 *
 *
 *
 *
 *
 *
 *
 *
in edited
book
thesis
 *
 *
 *
 *
 *
16.3. General formatting rules
• Article titles are printed in roman, with no quotation marks around them.
• Publication titles (both book titles and journal titles) are printed in italics.
• Editors are followed by (ed.) or (eds.) (depending on the number of editors).
• The author list, the year number, the article title, the editor list, the volume number,
the page numbers, and the publisher are followed by a period (as seen in the headings
of Table 1).
• The city is followed by a colon.
• Additional nonstandard parts may follow the reference in parentheses.
16.4. Standard reference types
Here are examples of the four standard types of references: journal articles, books, articles
in edited volume, and thesis:
• Journal article (journal title is immediately followed by the journal volume number):
Milewski, Tadeusz. 1951. The conception of the word in languages of North American natives.
Lingua Posnaniensis 3. 248–268.
• Book (whether authored or edited, book title followed by a period):
Matthews, Peter. 1974. Morphology. Cambridge: Cambridge University Press.
Lightfoot, David W. (ed.). 2002. Syntactic effects of morphological change. Oxford: Oxford University
Press.
• Article in edited volume (editor list is preceded by In and followed by (ed.) or (eds.)
and comma, book title is followed by a comma):17
Erdal, Marcel. 2007. Group inflexion, morphological ellipsis, affix suspension, clitic sharing. In
Fernandez-Vest, M. M. Jocelyne (ed.), Combat pour les langues du monde: Hommage à Claude
	
 	
 	
 	
 	
 	
 	
 	
 	
 	
 	
 	
 	
 	
 	
 	
 	
 	
 	
 	
 	
 	
 	
 	
 	
 	
 	
 	
 	
 	
 	
 	
 	
 	
 	
 	
 	
 	
 	
 	
 	
 	
 	
 	
 	
 	
 	
 	
 	
 	
 	
 	
 	
 	
 	
	
 
17
The complete information about the volume is always included, even if other articles from the same volume
are listed in the references. There is no need to list the volume itself separately, unless it is cited separately.
(This means that more space is needed, but it is otherwise much simpler than the old paper-saving
convention of making some references sensitive to the existrence of other references in the list).
	
 
 12	
  
Hagège, 177–189. Paris: L’Harmattan.
• Thesis (university is treated as publisher, type of thesis/dissertation is mentioned in
parentheses as a nonstandard part):18
Yu, Alan C. L. 2003. The morphology(Doctoral dissertation.)
andphonology of infixation. Berkeley: University of California.
Other kinds of publications should be treated like one of these to the extent that this is
possible. For example, published conference papers can be treated like articles in edited
volumes or like journal articles. In unpublished conference papers, the conference is treated
as a nonstandard part in parentheses (but such unpublished papers should only be cited
from recent conferences, if it can be expected that the material will eventually be
published):
Filppula, Markku. 2013. Areal and typological distributions of features as evidence for language
contacts in Western Europe. (Paper presented at the conference of the Societas Linguistica
Europaea, Split, 18–21 September 2013.)
16.5. Optional parts
Optionally, the journal volume number may be followed by an issue number, given in
parentheses:
Coseriu, Eugenio. 1964. Pour uneet de littérature 2(1). 139–186.
sémantique diachronique structurale.Travaux de linguistique
The book title mayin parentheses:
be followed by series information (series title plus series number), given
Lahiri, Aditi (ed.). 2000. Analogy, leveling, markedness: Principles of change in phonology and
morphology (Trends in Linguistics 127). Berlin: Mouton de Gruyter.
Series titles have special capitalization, like journal titles (see §2).
Especially issue numbers are very useful for retrieving articles, so it is recommended that
they should be used.
16.6. Author surnames and given names
The author names always appear in the order “surname, given name” in the list of
references,19 in order to make it unambiguously clear which elements of the author name
belong to the surname and which belong to the given name. If the second name in the
following example were given in the order “given name surname” (Francisco José Ruiz de
Mendoza), the parsing would not be clear.
	
 	
 	
 	
 	
 	
 	
 	
 	
 	
 	
 	
 	
 	
 	
 	
 	
 	
 	
 	
 	
 	
 	
 	
 	
 	
 	
 	
 	
 	
 	
 	
 	
 	
 	
 	
 	
 	
 	
 	
 	
 	
 	
 	
 	
 	
 	
 	
 	
 	
 	
 	
 	
 	
 	
	
 
18
The 2007 Unified Style Sheet has the university and the dissertation information as one single part, even
though they are quite different types of information (“Berkeley: University of California dissertation”).
19
This is a simplification over the 2007 Unified Style Sheet, which treats non-first names in author and
editor lists in a special way, with inverted order.
	
 
 13	
  
Pérez Hernández, Lorena \& Ruiz de Mendoza, Francisco José. 2002. Grounding, semantic
motivation, and conceptual interaction in indirect directive speech acts. Journal of
Pragmatics 34(3). 259–284.
When there are more than two authors (or editors), each pair of names is separated by an
ampersand.20 No author name is omitted, i.e. et al. is not used in references.
Chelliah, Shobhana \& de Reuse, Willem. 2010. Handbook of descriptive linguistic fieldwork.
Dordrecht: Springer.
Johnson, Kyle \& Baker, Mark \& Roberts, Ian. 1989. Passive arguments raised. Linguistic
Inquiry 20. 219–251.
21Surnames with internal complexity are never treated in a special way. Thus, Dutch or
German surnames that begin with van or von (e.g. van Riemsdijk) or French and Dutch
surnames that begin with with de (e.g. de Groot) are treated just like Belgian surnames (e.g.
De Schutter) and Italian surnames (e.g. Da Milano) and are alphabetized under the first
part, even though they begin with a lower-case letter. Thus, the following names are sorted
alphabetically as indicated (i.e. mechanically).
Da Milano, Federica > de Groot, Casper > De Schutter, Georges > de Saussure, Ferdinand > van der
Auwera, Johan > Van Langendonck, Willy > van Riemsdijk, Henk > von Humboldt, Wilhelm
When they occur in the prose text, they are not treated in a special way either, i.e. they
have lower case unless they occur at the begining of a sentence (this is in line with the
French and German practice,22 but in contrast to the Dutch practice), e.g.
as has been claimed by van Riemsdijk \& Williams (1981)
Chinese and Korean names may be treated in a special way: As the surnames are often not
very distinctive, the full name may be given in the in-text citation, e.g.
the neutral negation bù is compatible with stative and activity verbs (cf. Teng Shou-hsin 1973; Hsieh
Miao-Ling 2001; Lin Jo-wang 2003)
16.7. Internet publications
Regular publications that are available online are not treated in a special way, as this applies
to more and more publications anyway.
When citing a web resource that is not a regular scientific publication, this should be
treated like a book, to the extent that this is possible, e.g.
	
 	
 	
 	
 	
 	
 	
 	
 	
 	
 	
 	
 	
 	
 	
 	
 	
 	
 	
 	
 	
 	
 	
 	
 	
 	
 	
 	
 	
 	
 	
 	
 	
 	
 	
 	
 	
 	
 	
 	
 	
 	
 	
 	
 	
 	
 	
 	
 	
 	
 	
 	
 	
 	
 	
	
 
20
This is a simplification over the 2007 Unified Style Sheet, which treats the last pair of names differently
from the non-last pairs.
21
This is a simplification over the 2007 Unified Style Sheet, which treats “names with von, van, de, etc.” in a
special way.
22
With classical authors such as de Saussure and von Humboldt, the first part of the name can be (and is often)
omitted. But this is not possible with modern names (e.g. von Heusinger, never *Heusinger).
	
 
 14	
  
Native Languages of the Americas. 1998–2014. Vocabulary in Native American languages: Salish words.
http://www.native-languages.org/salish\_words.htm. (Accessed 2014-08-13.)
16.8. Miscellaneous
Books may include a volume number, separated from the book title by a comma:
Rissanen, Matti. 1999. Syntax. In Lass, Roger (ed.), Cambridge history of the English language,
vol. 3, 187–331. Cambridge: Cambridge University Press.
And there may be information about the edition, following the book title:
Croft, William.Press.
2003. Typology and universals. 2nd edn. Cambridge: Cambridge University
If a publisher is associated with several cities, only the first one needs to be given,
e.g. Berlin: De Gruyter Mouton, or Amsterdam: Benjamins.
Other nonstandard types of information may follow the standard parts in
parentheses, e.g.
Mayerthaler, Willi. 1988. Morphological naturalness. Ann Arbor: Karoma. (Translation of
Mayerthaler 1981.)
Titles of works written in a language that readers cannot be expected to know may be
accompanied by a translation, given in brackets:
Haga, Yasushi. 1998. Nihongo no shakai shinri [Social psychology in the Japanese language].
Tokyo: Ningen no Kagaku Sha.
Li, Rulong. 1999. Minnan fangyan de daici [Demonstrative and personal pronouns in Southern
Min]. In Li, Rulong \& Chang, Song-Hing (eds). Daici [Demonstrative and personal
pronouns], 263–287. Guangzhou: Ji’nan University Press.
If the title is not only in a different language, but also in a different script, it may be given
in the original script, in addition to the transliteration (following it in parentheses).
Likewise, the name of the author may be given in the original script, as follows:
Plungian, Vladimir A. (Плунгян, Владимир А.) 2000. Obščaja morfologija: Vvedenie v problematiku
(Общая морфология: Введение в проблематику) [General morphology: Introduction to the
issues]. Moskva: URSS.
Chen, Shu-chuan (陳淑娟). 2013. Taibei Shezi fangyan de yuyin bianyi yu bianhua (台北社子方言
的語音變異與變化) [The sound variation and change of Shezi dialect in Taipei city]. Language
and Linguistics 14(2). 371–408.
17. Rules for monographs and edited volumes
17.1. Parts of books
Books consist of the following parts (with optional parts in parentheses): title page,
colophon page, (dedication page,) contents, (acknowledgements or preface, abbreviations or
	
 
 15	
  
notation,) chapter 1, chapter 2, etc., appendix A, appendix B, etc., bibliography, name
index, (language index,), subject index.
Books may also group the chapters into parts (Part I, Part II, etc). A new part does not
start a new chapter numbering. Parts mainly serve to provide orientation in the table of
contents.
17.2. Monographs vs. edited books
Chapters of edited books are preceded by an abstract, like journal articles, but chapters of
monographs are not accompanied by an abstract.
Edited books are treated like a collection of journal articles, i.e. each article has its own
list of references and abbreviations, so that the articles can be read and understood
independently.
Chapters in edited volumes are numbered like chapters in monographs, but the chapter
number is not contained in the section number, i.e. Section 2 of Chapter 5 is §2, not §5.2.
17.3. Table of contents
The table of contents (called Contents) lists the chapters, chapter sections and subsections
(indented and preceded by their numbers), together with the page numbers.
17.4. Cross-references
While articles refer only to sections within the same article, books may refer to chapters
and sections within the same book, and to sections within the same chapter. Note that
when referring to parts of a book, §2.3 means §3 of Chapter 2.
17.5. Numbering tables and figures
In monographs, the numbers of tables and figures are preceded by the chapter number.
Thus, the second table in Chapter 3 is Table 3.2. (However, examples simply start with (1)
in each new chapter.)
This rule does not apply to chapters in edited volumes, as the chapter numbers are not
salient here.
17.6. Bibliographical references
When a self-standing chapter in an edited book contains a reference to another chapter in
the same book, the referred-to chapter is listed in the references in the normal way, as if it
were published in a different edited volume. However, the in-text citation may contain the
additional comment (in this volume) in parentheses, e.g.
As explained by Li \& Kim (2015) (in this volume), it is often useful to...
