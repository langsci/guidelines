%% -*- coding:utf-8 -*-
\chapter{Structure of books}

\section{Front matter}

The front matter of \lsp books is structured as follows
\begin{table}[h]
\begin{tabular}{p{4cm}p{6cm}}
 dedication & optional\\
table of contents & obligatory\\
notational conventions & optional\\
acknowledgements & optional\\
preface & optional\\
list of abbreviations & optional\\
\multicolumn{2}{l}{No lists of figures or tables!}\\
\end{tabular}
\end{table}
\section{Back matter}
The back matter is structured as follows:
\begin{table}[h]
\begin{tabular}{p{4cm}p{6cm}}
 Appendix A & optional \\
 Appendix B & optional \\
 further appendices & optional \\
 Bibliography & obligatory \\
 Author index & obligatory \\
 Language index  & optional (advisable if the book talks about a larger number of languages)\\
 Subject index & obligatory \\
\end{tabular}
 \end{table}


\chapter{Style rules}
\section{Generic rules}
We use the \em Generic Style Rules for Linguistics \em available on \url{http://www.eva.mpg.de/lingua/pdf/GenericStyleRules.pdf}

\section{House Rules}\label{sec:houserules}
The generic rules are complemented by the following house rules:

\subsection{Academic \emph{we}}

Monographs and articles that are authored by a single author should use the pronoun \emph{I} rather
than \emph{we} as in ``As I have shown in Section~3''.	
 

\subsection{British vs.\ American English}
Choose one and be consistent. For edited volumes, the choice is per chapter.  


\subsection{Figures}  
Please mention the creator and the licensing status of all photographs and all drawings in the caption unless they were created by you. The licensing must be compatible with the license chosen for the book. It is for instance legally not possible to include restricted copyrighted material in an open access book. 

Photographs should be in jpg format. For all drawings, maps, diagrams, etc., a vector format (*svg, *eps, *ps, *pdf) is preferred, *png is acceptable. For *jpg and *png, resolution should be at least 320 dpi. 

See \url{http://userblogs.fu-berlin.de/langsci-press/2016/12/12/graphics-and-images/} for more information.

\subsection{Tables}
Be aware that the book will not be printed on A4/letter paper. Our paper is 17cm x 24cm; A4 is 21cm x 29.4cm.  This means that you have less horizontal and vertical space for your tables. Tables should generally not have more than about 6 columns and about 10 rows (\tabref{tab:sentencetable}). If the data in the cells are very, short like numbers, phonemes or very short words for instance, there might be exceptions (\tabref{tab:lexicaltable}).



\begin{table}
\begin{tabularx}{\textwidth}{X>{\raggedright}XXXXX}
\lsptoprule
 & English & German & French & Spanish & Italian  \\
\midrule
1 & We wish you a merry christmas & Fröhliche Weihnachten & Joyeux Noël & Feliz Navidad  & Buon natale \\
2 & How is the weather today?& & &  & \\
3 & We appreciate your business& & &  & \\
4 & I do not want to buy this carpet& & &  & \\
5 & Please circulate& & &  & \\
6 & Apologies for cross-postings& & &  & \\
7 & Colorless green ideas sleep furiously& Kleine grüne Ideen schlafen wütend & &  & \\
8 & The man hit the woman and ran away& & &  & \\
9 & John gives Mary the book  & & &  & \\
10 & repeat ad libitum  & & & & \\
\lspbottomrule
\end{tabularx}
\caption{Maximal number of colums and rows in a typical table with sentences.}
\label{tab:sentencetable}
\end{table}


\begin{table}
\begin{tabularx}{\textwidth}{XXXXXX}
\lsptoprule
 & English & German & French & Spanish & Italian  \\
\midrule
1 & one  & eins & un & uno  & uno \\
2 & two & zwei& deux& dos  &  due\\
3 & three& & &  & \\
4 & four & & &  & \\
5 & five & & &  & \\
6 & & & &  & \\
7 & & & &  & \\
8 & & & &  & \\
9 & & & &  & \\
10 & & & & & \\
11 & & & & & \\
12 & & & & & \\
13 & & & & &  \\
14 & & & & &  \\
15 & & & & &  \\
16 & & & & &  \\
17 & & & & &  \\
18 & & & & &  \\
19 & & & & &  \\
20 & & & & &  \\
21 & & & & &  \\
22 & & & & &  \\
23 & & & & &  \\
24 & & & & &  \\
25 & & & & &  \\
26 & & & & &  \\
27 & & & & &  \\
28 & & & & &  \\
29 & & & & &  \\
30 & & & & &  \\
31 & & & & &  \\
32 & & & & &  \\
33 & & & & &  \\
34 & & & & &  \\
35 & & & & &  \\ 
\lspbottomrule
\end{tabularx}
\caption{Maximal number of colums and rows in a typical lexical table.}
\label{tab:lexicaltable}
\end{table}


All tables should fit on one page. It is not permitted to break the page in the middle of a table. If your content is very long, split the large table into several smaller ones.


\subsection{Abbreviations}
If you need special abbreviations that are not defined by the Leipzig Glossing Rules, put them in a table in a special section with abbreviations immediately before the first chapter of a monograph. In the case of an edited volume, the lists of abbreviations should be placed immediately before the references of the individual chapters.

\subsection{Glossed examples}
The formatting of example sentences in the typological series follows the format that is used by the World Atlas of Language Structures \citep{WALS}: If there is just one example sentence for an example number, the language name follows the example number directly, as in (\ref{ex-typology}); it may be followed by the reference.

{\def\exfont{\normalsize\itshape}
\ea\label{ex-typology}
\langinfo{Mising}{}{\citealt[69]{Prasad91a}}\\
\gll azɔ́në dɔ́luŋ\\
     small village\\ 
\glt `a small village' 
\z


If there are two sub-examples for a single example number, the example heading may have scope over both of them:

\ea
\langinfo{Zulu}{}{Poulos \& Bosch 1997: 19; 63}\\
\ea
\gll Shay-a		inja!\\
hit-\textsc{imp.2sg}	dog\\
\glt `Hit the dog!'
\ex
\gll	Mus-a	uku-shay-a	inga! \\
	\textsc{neg.imp.aux-2sg}	\textsc{inf}-hit-\textsc{inf}	dog \\
\glt		`Do not hit the dog!'	
\z
\z

% If two examples with different numbers belong to the same language, the language name is repeated only if the identity of the language is not clear from the context. 
If an example consists of several sub-examples from different languages, the language name and references follow the letters, as in (\ref{ex:apatani}).

\ea\label{ex:apatani}
\ea
\langinfo{Apatani}{}{\citealt[23]{Abraham85a}}\\
\gll aki atu\\ 
     dog small\\ 
\glt ‘the small dog’ 
\ex 
\langinfo{Temiar}{}{\citealt[155]{Benjamin76a}}\\ 
\gll dēk mənūʔ\\
     house big\\
\glt ‘big house’ 
\z
\z

You should use the numbered example environment only for linguistic examples, theorems and the like. Lists of consultants, lists of recordings, lists of geographical places where a language is spoken and the like should be put in a table environment. 


\subsection{Quotations}

If long passages are quoted, they should be indented and the quote should be followed by the exact reference. Use the quotation environment \latex provides:
\begin{quote}
Precisely constructed models for linguistic structure can play an
important role, both negative and positive, in the process of discovery 
itself. By pushing a precise but inadequate formulation to
an unacceptable conclusion, we can often expose the exact source
of this inadequacy and, consequently, gain a deeper understanding
of the linguistic data.
\citep[5]{Chomsky57a}
\end{quote}
%
Short passages should be quoted inline using quotes: \citet[5]{Chomsky57a} stated that ``[o]bscure
  and intuition-bound notions can neither lead to absurd conclusions nor provide new and
correct ones''.

If you quote text that is not in the language of the book provide a translation. Short quotes should
be translated inline, long quotes should be translated in a footnote.




\subsection{Cross-references in the text}

Please use the cross-referencing mechanisms of your text editing/type setting software. Using such
cross-referencing mechanisms is less error-prone when you shift text blocks around and in addition
all these cross-references will be turned into hyperlinks between document parts, which makes the
final documents much more useful.

% If you have numbered example sentence, please start with (1) for every new chapter.
%%
%% Alternatively, you may put the chapter number in front of the example number (thus starting with (7.1), (7.2), ... in chapter 7, for example).
%% [COMMENT: Bei Grammatiken ist es durchaus üblich, dass alle Beispiele im Buch durchnummeriert werden. Auch in Rießlers Arbeit sind die Beispiele komplett durchnummeriert. Ich bin mir nicht sicher, wie wichtig diese Regelung ist, und ob wir nicht Volldurchnummerierung auch erlauben sollten.]

 
Depending on the series and the language the book is published in authors may  use the § sign or the word \emph{Section}. 

\subsection{Epigrams}
You can use epigrams for your chapters. When using epigrams in edited volumes, make sure that the combination of epigram and abstract leaves room for the actual chapter to start on the same page.

\subsection{Aspiration, labialization, velarization, etc.}
For phonetic symbols of aspiration and secondary articulation, do not use a superscript normal letter; use the special Unicode characters ʰ ʷ ʲ ˠ ˤ etc. You can define a special command like \verb+{\lab}+ for ʷ for easier input. 

\subsection{Footnotes in section titles}\label{sec:footnote}
You should not use footnotes in section titles.\footnote{The footnote about \sectref{sec:footnote} can easily be added after the first sentence of the running text.} Very often, a plain sentence in the running text will be just as good. If you really want a footnote, insert it after the first sentence of the relevant section.

\subsection{Color}
Use color sparingly. Color should never be the only means to access information, but can be used as an addition. There are two reasons for this: b/w printers and colorblind people. Your work will not be accessible in those cases if color is the only distinguishing feature. Alternatives to color are using different shapes or shadings instead of color, grouping items, or verbal description. 




\section{Citations and references}
\label{sec-references-authors}

Please deliver a \bibtex file with all your references together with your submissions. 
\bibtex can be exported from all common bibliography tools (We recommend BibDesk for the Mac and JabRef for all other platforms). 

Please provide all first and last names of all authors and editors. Do not use {\em et~al.}  in the \bibtex file; this will be generated automatically when inserted.

For bipartite family names like ``von Stechow'', ``Van Eynde'', and ``de Hoop'' make sure that these
family names are contained in curly brackets.
% These authors will then be cited as \citet{VanEynde2006a} and \citet{vonStechow84a}.
 Note that Dutch names like ``de Hoop'' are not treated differently from other surnames.

Many bibliographies have inconsistent capitalization. We do not use Title Case, i.e. all words are spelled as they would be spelled in running text (sentence case). Hence, we use \em A grammar of Tagalog \em and not \em A Grammar of Tagalog\em. There are two strategies to achieve this: decapitalization with protection, or literal output. If you use decapitalization with protection, all titles and booktitles will be decapitalized  If there is a proper name in a title, enclose it in \verb+{}+ to prevent decapitalization, e.g. \verb+title = {The languages of {A}frica}+. Use the same procedure for German nouns and all other characters in titles which should not be decapitalized. This is not necessary for other fields, especially the author and editor fields, where capitalization is kept as is. You can use your *bib file for publisher requiring title case and for publishers requiring sentence case.

If you opt for literal output, add the option \verb+undecapitalize+ to \verb+main.tex+. In this case, it is your task to lowercase all words in Title Case in your bibliography. You cannot use your *bib file for publisher which require Title Case anymore. 

The references in your \bibtex file will automatically be typeset correctly. So, provided the
\bibtex file is correct, authors do not have to worry about this. But there are some things to
observe in the main text. Please cite as shown in Table~\ref{tab-citation}.

\begin{table}[bt] 
\caption{Citation style for \lsp}%
\label{tab-citation}
\begin{tabular}{p{1.2cm}>{\small}p{6.2cm}p{3.8cm}}
\lsptoprule
citation type & example &yields\\
\midrule
author & \raggedright \verb+As \citet[215]{MZ85a}+
	  \verb+have shown+             &As \citet[215]{MZ85a} have shown\\ 
       & \raggedright  \verb+As \citet[215]{MZ85a} and+
	  \verb+\citet{Bloomfield1933lg}+
	  \verb+have shown+ 
					  &  As \citet[215]{MZ85a} and \citet{Bloomfield1933lg} have shown\\ 
work   & \raggedright  \verb+As was shown in+ 
	  \verb+\citet[215]{Saussure16a},+
	  \verb+this is a problem for theories that ...+ & As was shown in \citet[215]{Saussure16a}, this is a problem for theories that \ldots\\ 
work   & \raggedright  \verb+This is not true \citep{+  
	  \verb+Saussure16a,Bloomfield1933lg}.+ & This is not true \citep{Saussure16a,Bloomfield1933lg}.\\[2em]
no double parentheses   & \raggedright \verb+This is not true+
			    \verb+(\citealt{Saussure16a}+
			    \verb+and especially+
			    \verb+\citealt{Bloomfield1933lg}).+& This is not true (\citealt{Saussure16a} and especially \citealt{Bloomfield1933lg}).\\
\lspbottomrule
\end{tabular}

\end{table}
\nocite{Bresnan82b}% something with an editor.
 
If you have an enumeration of references in the text as in \emph{As X, Y, and Z have shown}, please use
the normal punctuation of the respective language rather than special markup like `;'.

% If you refer to regions in a text, for instance 111--112, please do not use 111f.\ or 111ff.\ but provide the
% full information.  

\section{Indexes}
All {\lsp} books have a Subject Index and a Name Index. The Language Index is optional and should be used if the book treats several languages. Subject Index and Language Index have to be prepared by the authors. We can automatize some of this if you send us a list of languages and a list of subject terms. You might want to try \url{sketchengine.co.uk} to compile a candidate list. 

The Name Index is generated automatically from the citations in the text. This means that you only have to add people to the Name Index who, for whatever reason, are mentioned without connection to a work in the list of references. Examples would be politicians, ancient philosophers, novelists and the like.
 