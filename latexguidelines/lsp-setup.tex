%% -*- coding:utf-8 -*-
\chapter{General information on Language Science Press}


\section{Background and motivation}

Language Science Press is a book imprint that publishes high-quality books in the field of academic
linguistics. It was founded in 2013, growing out of the initiative ``Open-Access Books for
Linguistics'' (OALI) that was started by Stefan Müller (and other linguists at FU Berlin) and joined
by Martin Haspelmath. After its first launch in August 2012, it quickly found several hundred
supporters from various subfields of linguistics and a range of different countries, including some
very prominent linguists.

The problem to which this initiative responded was the increasing cost of linguistics books, which
is in increasingly stark contrast with the ease with which files can be shared
\citep{MuellerOA}. More and more, it seems that most of what the traditional publishers add to the
scientists' work is the prestige of an imprint label \citep{Haspelmath2012a}, but this is something
that is ultimately created by the scientists as well.

Thus, we decided to found a new imprint (\lsp) dedicated to publishing high-quality books which exist primarily in electronic form. Printed copies will be available through print-on-demand services. This imprint will be owned and run by scholars, and neither authors nor readers will be charged. The required work (reviewing, proofreading, typesetting) will be organized and carried out by the scholars themselves.

Language Science Press is associated with the FU Berlin and is directed by Stefan Müller and Martin Haspelmath.

In December 2013, the DFG (Deutsche Forschungsgemeinschaft) awarded us a substantial amount of funding, which allows us to employ a number of people to develop our activities in various domains.


%% \section{Set Up and Responsibilities}

%% Language Science Press works with Open Monograph Press’s (OMP) management software.
%% (more details will follow later)

\section{Strategy}

We are working with the assumption that book publication can nowadays be organized in a much cheaper and more efficient way. Essentially, to publish high-quality books, the following tasks need to be carried out:

\begin{enumerate}
\item[(i)] manuscript reviewing
\item[(ii)] typesetting
\item[(iii)] proofreading
\item[(iv)] overall coordination
\item[(v)] hosting
\end{enumerate}

We are assuming that (i) can be done by the series editors without much help from the coordinators,
that (ii) can be done by the authors (initially with help from the coordinators and later with help
from the series editors), that (iii) can be done by volunteers from the LangSci community, and that
the work for (iv) will get less and less as we develop a routine. The fifth point, hosting, is taken
care of by the FU Berlin.

The main challenges are (ii) typesetting by the authors and (iv) making the coordination tasks
slim. Professional typesetting requires the use of LaTeX, and while an increasing number of
linguists is familiar with this typesetting software, many others are not. But we trust that there
will be a sufficient number of linguists willing to invest the effort to do the LaTeX typesetting
(or to find someone to do it for them). This Guidelines text provides the necessary information
about \latex classes.


To reduce the amount of coordination that is required, series editors and authors will have to conform very strictly to our standard procedures. While commercial publishers with permanent staff members can afford to allow deviations from the general rules, this is not really possible with our model. Authors and editors who find our procedures too inflexible will have to choose alternative publication outlets. The present Guidelines set out the rules that editors and authors must obey if they want to publish with Language Science Press.



\section{Responsibilities}

All books published by Language Science Press appear in book series, which are managed by a Series
Editor (or a team of Editors). The Series Editors are in charge of the reviewing and the coordination of the production of
the books in their series. The overall coordination of the Press is in the hands of the Press
Directors Stefan Müller and Martin Haspelmath.

\subsection{Advisory board}

The Advisory Board was particularly important in the early stage of Language Science Press, when there were few series. Its task was and is to assess proposals for new series, as well as to give respectability to the whole enterprise.


\subsection{Series and editorial boards}

Each series is run by a team of Series Editors, who bear full responsibility for manuscript
reviewing, selection and coordination of production. The Series Editors are generally supported by
an Editorial Board, i.\,e.\ 5--25 colleagues from various places whose expertise falls in the area of
the series. Editorial Board members should be willing to review at least one book manuscript per
year.

\subsection{Open Monograph Press and CEDIS}

Language Science Press uses the Public Knowledge Project’s software \emph{Open Monograph Press} (OMP),
which was specifically designed for open-access publishers. We are in regular contact with OMP's
software developers.

The OMP software is hosted by the CeDiS, who also provides support for authors and editors.


\subsection{The library of the Freie Universität Berlin}

The library of the Freie Universität Berlin is in charge of storing the published versions of the
books on the document server of the Freie Universität and i sresponsible for providing
bibliographical metadata for the books. 

%% \subsection{Design}

%% The design of the Language Science Press books (text design and cover design) has been done by professional designer Ulrike Harbort.


\section{Open access and licence}

All Language Science Press books are published with open access, i.\,e.\ they can be downloaded free of
charge. All rights (copyrights, translation rights) remain with the author. 

By default, \lsp books are published with a Creative Commons CC-BY licence\footnote{ 
  Currently \url{http://creativecommons.org/licenses/by/4.0/}, 16.02.2014.
} (see \citew{Shieber2012a} for details
of what this means and why it is the preferred licence for scientific papers and books). The CC-BY
license allows for free reuse of the material in the book, including commercial uses as for instance
edited volumes that contain parts of the book licensed under CC-BY. The only condition is that the
work is properly attributed to the author/authors. The CC-BY license guarantees maximal distribution
of the material.

In certain situations, a CC-BY license is not possible. For instance if \lsp publishes a translation of a
book that already appeared with another publisher. In such situations the books will be published
under the more restrictive CC-BY-ND license\footnote{
Currently \url{http://creativecommons.org/licenses/by-nd/4.0/}, 16.02.2014.
}, which forbids to change the material (NoDerivatives) and hence guarantees that
the rights of the original publisher are not violated by somebody translating the work back into the
original language and distributing the book commercially or non-commercially.


%(more details will follow later)


\section{Print on demand}

There will be a print-on-demand service connected to the Language Science Press website. Thus, it will also possible to purchase printed copies of the books.



%      <!-- Local IspellDict: en_US-w_accents -->
