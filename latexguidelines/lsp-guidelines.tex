%% -*- coding:utf-8 -*-
\documentclass[ number=??
                ,series=lnls,
                ,isbn=000-0-000000-00-0,
                ,url=http://langsci-press.org/catalog/book/0,
	        ,output=long    % long|short|inprep              
	        %,blackandwhite
	        %,smallfont
	        ,draftmode  
		  ]{LSP/langsci}                          

          
\hypersetup{pdfdisplaydoctitle=true}

\usepackage{xspace}
\newcommand{\latex}{\LaTeX\xspace}


\usepackage{embrac}

% just for the XeLaTeX logo
\usepackage{dtklogos}\newcommand{\xelatex}{\XeLaTeX\xspace}
\newcommand{\bibtex}{\BibTeX\xspace}

\usepackage{LSP/lsp-styles/avm}

\usepackage{lsp-makros}

\usepackage{german}\selectlanguage{USenglish}

\usepackage{LSP/lsp-styles/lsp-eng-hyp}

\usepackage{styles/abbrev}


      
% OT pointing hand
\usepackage{pifont}
\newcommand{\hand}{\ding{43}}

% OT tableaux                                                
\usepackage{pstricks,colortab}   

% DRS package by Alexis Dimitriadis
\usepackage{drs}

\usepackage{tabularx}

% AVMs
\usepackage{avm}
\avmfont{\sc} 
\avmvalfont{\it}

% command to fontify the type values of an avm 
\newcommand{\tpv}[1]{{\avmjvalfont #1}}

% command to fontify the type of an avm and avmspan it
\newcommand{\tp}[1]{\avmspan{\tpv{#1}}}


\usepackage{tikz-qtree}
% has strange side effects
%\tikzset{every tree node/.style={align=left, anchor=north}}
\tikzset{every roof node/.append style={inner sep=0.1pt,text height=2ex,text depth=0.3ex}}

\usepackage{jambox}


\setlength{\marginparwidth}{1.5cm}
\usepackage[textsize=tiny,textwidth=1.5cm]{todonotes}
\newcommand{\todostefan}[1]{\todo[color=green!40]{#1\xspace}}
\newcommand{\inlinetodostefan}[1]{\todo[color=green!40,inline]{{\normalsize #1}}}


% Chinese
\usepackage[indentfirst=false]{xeCJK}
\setCJKmainfont{SimSun}

% bidirectional text and support for Arabic/Persian
\newfontfamily\Parsifont[Script=Arabic]{XB Niloofar}
%\usepackage{bidi}
\usepackage{LSP/lsp-styles/lsp-bidi}
\newcommand{\PRL}[1]{\RL{\Parsifont #1}}
%\TeXXeTOff
\usepackage{LSP/lsp-styles/lsp-gb4e}

\def\exfont{\normalsize\itshape}

%% \iflsDraft
%% \proofmodetrue
%% \fi

% This is needed to allow for hyphenation in \texttt
% http://tex.stackexchange.com/questions/44361/how-to-automatically-hyphenate-within-texttt
% usually this is not required, but since we show index entries for packages in the margin and since
% the package names a re typeset in tt, we need the hyphenation.
\DeclareFontFamily{OT1}{cmtt}{\hyphenchar \font=-1}


\title{Language Science Press guidelines}  
\subtitle{General rules for editors, authors and \latex recommendations}
\BackTitle{Language Science Press guidelines}

\BackBody{This book contains the guidelines for Language Science Press authors and editors. For those who want
  to help keeping the production costs low and therefore decided to use \latex, it also contains
  descriptions of packages that can be used for typesetting trees, Attribute Value Matrices, OT-tableaux, Categorial
  Grammar proofs, LFG analyses, and much more. The setup of typesetting script with special fonts as
  for instance right to left scripts like Arabic is explained. The \latex chapter also contains sections
  concerning the efficient workflow in professional typesetting environments using \latex.

Stefan Müller is an experienced \latex user who has typeset four published books and several book
manuscripts and journal articles.
}

\dedication{This book is dedicated to everybody who cannot afford to buy books by profit oriented publishers.}

\typesetter{Stefan Müller}
\proofreader{Test T.\ Tester, Test T.\ Tester, Test T.\ Tester, Test T.\ Tester, Test T.\ Tester, Test T.\ Tester,}

\author{Stefan Müller and Martin Haspelmath}



\makeatletter
\def\verbatim@font{\scriptsize\ttfamily}
\makeatother
         
%\renewcommand{\eachwordone}{\it}
%\renewcommand{\exfont}{\it}
%\def\exfont{\it}

\begin{document}               
         
                                                                           
                                  
\maketitle                

%\frontmatter

\chapter*{Preface}

%\lipsum[3-10]  

This book has several purposes: it describes the editorial process and contains guidelines with some style rules for all
authors. In addition it contains a part for authors who use \latex or who want to learn \latex in order to
support \lsp. The \latex part is also a reference for those who volunteered to help typesetting
manuscripts that were not submitted in \latex. See \citew{MuellerOA} and \citew{MH2013a} for an overview of the
general setup of the project.

\section*{Acknowledgements}



%% David Reitter danke ich für die \LaTeX"=Makros für Combinatorial
%% Categorial Grammar, Mary Dalrymple und Jonas Kuhn für die LFG"=Makros und Beispielstrukturen, und
%% Laura Kallmeyer für die \LaTeX"=Quelltexte der meisten TAG"=Analysen. Ohne diese
%% Originalmakros/-texte wären die Darstellungen in den jeweiligen Kapiteln sicher nicht halb so schön
%% geworden. 

This book is typeset with \xelatex. We thank the \latex developers for their work and the members of
the \textit{German Language TeX Users Group Communication List} and those replying at \url{http://tex.stackexchange.com} for many usefull hints and suggestions.

We thank Matthias Hüning\aimention{Matthias H{\"u}ning} for comments on an earlier version of this document and Corinna Handschuh\aimention{Corinna Handschuh}
and Francesco Cangemi\aimention{Francesco Cangemi} for being the first to use the new \latex classes and providing feedback
to us.

\bigskip

\noindent
Berlin, \today\hfill Stefan Müller \& Martin Haspelmath


\tableofcontents      

\mainmatter         

%% -*- coding:utf-8 -*-
\chapter{General information on Language Science Press}


\section{Background and motivation}

Language Science Press is a book imprint that publishes high-quality books in the field of academic
linguistics. It was founded in 2013, growing out of the initiative ``Open-Access Books for
Linguistics'' (OALI) that was started by Stefan Müller (and other linguists at FU Berlin) and joined
by Martin Haspelmath. After its first launch in August 2012, it quickly found several hundred
supporters from various subfields of linguistics and a range of different countries, including some
very prominent linguists.

The problem to which this initiative responded was the increasing cost of linguistics books, which
is in increasingly stark contrast with the ease with which files can be shared
\citep{MuellerOA}. More and more, it seems that most of what the traditional publishers add to the
scientists' work is the prestige of an imprint label \citep{Haspelmath2012a}, but this is something
that is ultimately created by the scientists as well.

Thus, we decided to found a new imprint (\lsp) dedicated to publishing high-quality books which exist primarily in electronic form. Printed copies will be available through print-on-demand services. This imprint will be owned and run by scholars, and neither authors nor readers will be charged. The required work (reviewing, proofreading, typesetting) will be organized and carried out by the scholars themselves.

Language Science Press is associated with the FU Berlin and is directed by Stefan Müller and Martin Haspelmath.

In December 2013, the DFG (Deutsche Forschungsgemeinschaft) awarded us a substantial amount of funding, which allows us to employ a number of people to develop our activities in various domains.


%% \section{Set Up and Responsibilities}

%% Language Science Press works with Open Monograph Press’s (OMP) management software.
%% (more details will follow later)

\section{Strategy}

We are working with the assumption that book publication can nowadays be organized in a much cheaper and more efficient way. Essentially, to publish high-quality books, the following tasks need to be carried out:

\begin{enumerate}
\item[(i)] manuscript reviewing
\item[(ii)] typesetting
\item[(iii)] proofreading
\item[(iv)] overall coordination
\item[(v)] hosting
\end{enumerate}

We are assuming that (i) can be done by the series editors without much help from the coordinators,
that (ii) can be done by the authors (initially with help from the coordinators and later with help
from the series editors), that (iii) can be done by volunteers from the LangSci community, and that
the work for (iv) will get less and less as we develop a routine. The fifth point, hosting, is taken
care of by the FU Berlin.

The main challenges are (ii) typesetting by the authors and (iv) making the coordination tasks
slim. Professional typesetting requires the use of LaTeX, and while an increasing number of
linguists is familiar with this typesetting software, many others are not. But we trust that there
will be a sufficient number of linguists willing to invest the effort to do the LaTeX typesetting
(or to find someone to do it for them). This Guidelines text provides the necessary information
about \latex classes.


To reduce the amount of coordination that is required, series editors and authors will have to conform very strictly to our standard procedures. While commercial publishers with permanent staff members can afford to allow deviations from the general rules, this is not really possible with our model. Authors and editors who find our procedures too inflexible will have to choose alternative publication outlets. The present Guidelines set out the rules that editors and authors must obey if they want to publish with Language Science Press.



\section{Responsibilities}

All books published by Language Science Press appear in book series, which are managed by a Series
Editor (or a team of Editors). The Series Editors are in charge of the reviewing and the coordination of the production of
the books in their series. The overall coordination of the Press is in the hands of the Press
Directors Stefan Müller and Martin Haspelmath.

\subsection{Advisory board}

The Advisory Board was particularly important in the early stage of Language Science Press, when there were few series. Its task was and is to assess proposals for new series, as well as to give respectability to the whole enterprise.


\subsection{Series and editorial boards}

Each series is run by a team of Series Editors, who bear full responsibility for manuscript
reviewing, selection and coordination of production. The Series Editors are generally supported by
an Editorial Board, i.\,e.\ 5--25 colleagues from various places whose expertise falls in the area of
the series. Editorial Board members should be willing to review at least one book manuscript per
year.

\subsection{Open Monograph Press and CEDIS}

Language Science Press uses the Public Knowledge Project’s software \emph{Open Monograph Press} (OMP),
which was specifically designed for open-access publishers. We are in regular contact with OMP's
software developers.

The OMP software is hosted by the CeDiS, who also provides support for authors and editors.


\subsection{The library of the Freie Universität Berlin}

The library of the Freie Universität Berlin is in charge of storing the published versions of the
books on the document server of the Freie Universität and i sresponsible for providing
bibliographical metadata for the books. 

%% \subsection{Design}

%% The design of the Language Science Press books (text design and cover design) has been done by professional designer Ulrike Harbort.


\section{Open access and licence}

All Language Science Press books are published with open access, i.\,e.\ they can be downloaded free of
charge. All rights (copyrights, translation rights) remain with the author. 

By default, \lsp books are published with a Creative Commons CC-BY licence\footnote{ 
  Currently \url{http://creativecommons.org/licenses/by/4.0/}, 16.02.2014.
} (see \citew{Shieber2012a} for details
of what this means and why it is the preferred licence for scientific papers and books). The CC-BY
license allows for free reuse of the material in the book, including commercial uses as for instance
edited volumes that contain parts of the book licensed under CC-BY. The only condition is that the
work is properly attributed to the author/authors. The CC-BY license guarantees maximal distribution
of the material.

In certain situations, a CC-BY license is not possible. For instance if \lsp publishes a translation of a
book that already appeared with another publisher. In such situations the books will be published
under the more restrictive CC-BY-ND license\footnote{
Currently \url{http://creativecommons.org/licenses/by-nd/4.0/}, 16.02.2014.
}, which forbids to change the material (NoDerivatives) and hence guarantees that
the rights of the original publisher are not violated by somebody translating the work back into the
original language and distributing the book commercially or non-commercially.


%(more details will follow later)


\section{Print on demand}

There will be a print-on-demand service connected to the Language Science Press website. Thus, it will also possible to purchase printed copies of the books.



%      <!-- Local IspellDict: en_US-w_accents -->

%% -*- coding:utf-8 -*-

\chapter{Guidelines for editors}

\section{Decision structure}


Each \lsp series has a team of Series Editors, who decide which books are accepted for the
series. There can be up to three Series Editors per series; if more people are involved at the top
level, one or two have to be the Chief Editors, and the others are Consulting Editors (or simply
Editors).

In addition, each book series normally has an Editoral Board of 10--35 members. The Editorial Board
members advise the Series Editors in various ways concerning the series, in particular by writing
manuscript reviews. However, the list of names of the Editoral Board also serves to indicate the
kind of orientation that the series is inteded to take, and not least to give prestige to the
series. Editorial Board membership is normally for a period of three years (renewable).

For the first seven books in each series, acceptance is conditional on approval by the Press
Coordinators. This ensures that there is agreement between the Series Editors and the Press
Coordinators on the level of quality of the series. This is important to ensure a uniformly high
quality of all series.

\section{Series web pages}

Each series has a homepage, which lists the Series Editors, the Editorial Board members (with
affiliation), and contains an Aims and Scope statement.

All published books are listed on the page of the series. (They can also be found elsewhere on the
\lsp site, e.g. under ``\href{http://langsci-press.org/catalog}{Catalog}''.) This page may also list
forthcoming books, i.e. books which have been accepted, revised and approved and are at the
production stage.

As soon as a book has been accepted and approved, it can be put on the website as ``forthcoming'',
with the bibliographical information, but without the actual downloadable file. This will serve the
purpose of advance publicity.

(more details will follow later)

\section{Types of book manuscripts}

Language Science Press books may be monographs or edited volumes in English, German, French, Spanish,
and Portugese. Which languages are accepted depends on the particular series.

The manuscripts should have a size of at least 80 pages and at most 800 pages.
% We decided to remove the upper limit. IDS grammar has 2500 pages, what about dictionaries?
There are no technical reasons for excluding shorter and longer manuscripts, but such manuscripts
are not clearly within the scope of what readers would expect when they hear ``book''. Shorter works
are perhaps better published as journal articles, and longer works are difficult to organize a
serious reviewing process for.

(more details will follow later)


\section{Submission and reviewing procedure (monographs)}

Book manuscripts are officially submitted by entering them into the OMP system. Of course,
informal preliminary submission (by e-mail or by some file sharing mechanism) is
possible. Official submission implies that all Series Editors (as well as the Press Coordinators)
are informed of the submission, if the submission is done without OMP.


In a next step, the manuscript is made available to the reviewers via the OMP system (initially,
while not everyone is familiar with it, this can be done informally, e.g. by e-mail). For each book
manuscript, at least two reviews are solicited, within a time frame of two months. The reviews are
made available to the Press Coordinators. The Series Editors may override the recommendations of the
reviewers, but if all reviewers are mostly negative, this needs to be justified to the Press
Coordinators.

If a reviewer does not react even after three months, it is recommended that the Series Editors
solicit at least one additional review. If within six months after submission fewer than two reviews
are returned, the manuscript counts as rejected.

If a manuscript was rejected, the same author may submit another manuscript a year after the
submission of the rejected manuscript. The new manuscript may be similar to the originally submitted
manuscript, so the author may think of this as a ``resubmission''. However, there is no official
resubmission procedure in Language Science Press, and there is no ``revise and resubmit'' decision.

Note that Language Science Press does not issue ``contracts'' on the basis of book proposals, like
other publishers do. Book proposals may be discussed informally with the Series Editors, and the
Editors may informally encourage the author to submit a book on the basis of an informal book
proposal, but none of this has any binding status.

\section{Submission and reviewing procedure (edited volumes)}

For edited volumes, the Series Editors may adopt the same procedure as for monographs, or
alternatively they may accept the volume without review, i.e. they delegate the quality control to
the book editor. However, this is possible only if the papers underwent a comments \& revision
process, and if upon submission, the book editor gives a full account of the comments \& revision
procedure to the Series Editors. In such a case, a book manuscript may be accepted without revision.

\section{Acceptance}

On the basis of the reviewers' reports, the Series Editors decide whether the book is accepted for
the series or not.

If revisions are needed or recommended (as is likely to be the case), then this is a preliminary
acceptance, conditional on proper execution of revisions. However, peliminary acceptance means that
an author is allowed to cite the book as ``to appear with Language Science Press''.

Upon acceptance of a book manuscript, not only the author and the Press Coordinators, but also all
the other Series Editors are informed, so that they stay informed of developments within the entire
Press (see Section~\ref{sec-editors-information}).
%\todostefan{This should be done by OMP.}


\section{Revision}

If a book manuscript is accepted, the Series Editors convey the reviews and their own comments to
the author, and the author is asked to revise the manuscript.

The Series Editors may specify some Required Changes on which the definitive acceptance is
conditional. The Required Changes may only be highly specific changes that are not very
time-consuming. Vague proposals for changes (``the approach needs to be more firmly grounded in
theory'', etc.), or changes that require a lot of additional work, are not acceptable as Required Changes.

Apart from the Required Changes, authors may choose to ignore recommended changes, but these cases
need to be justified to the Series Editors. In the case of a serious disagreement between author and
Series Editors, the Press Coordinators are ready to mediate.

If the changes are made as requested the book will receive Definitive Acceptance.  The revision
stage includes proofreading. Like the revision of the content, this is the Series Editors'
responsibility, but the \lsp Community will be able to help with this. (Details will follow later.)

\section{Production}

Once the revised version of a manuscript has been returned by the author and Definitively Accepted
by the Series Editors, production can begin.

\subsection{Rough typesetting}

LaTeX styles are applied, figures are created in the proper way, etc.

\subsection{Formal contract}

At this stage, the author signs a contract with the FU Berlin (which is reponsible for hosting and permanent archiving) about the legal publication of the book. The contract form can be downloaded from the following page:

http://edocs.fu-berlin.de/docs/content/main/autoren/vertraege.xml?lang=en

Basically only the author's address needs to be filled in, as well as the book title and the URL (http://langsci-press.org/catalog/book/...).

This contract is necessary for the application for an ISBN number, which is needed for typesetting.

\subsection{Metadata and catalog}

The Series Editors/Authors enter the following metadata about the book into OMP:
\begin{itemize}
\item book synopsis (for the web page and back cover)
\item author bio
\item add keywords, regions, languages, and so on
\end{itemize}
The book also needs to be assigned to a category. At the moment, we are working with the following
categories:
\begin{itemize}
\item Phonetics and Phonology Phonetics
\begin{itemize}
\item Phonetics
\item Phonology
\end{itemize}
\item Morphology
\item Syntax
\item Semantics
\item Pragmatics
\item Historical Linguistics
\begin{itemize}
\item Comparative Historical Linguistics 
\end{itemize}
\item Typology
\end{itemize}
Once all these things have been taken care of, the book can be announced in the catalog as ``forthcoming''.

\subsection{Community proofreading/commenting}

(Details will follow later. Maybe at this stage the manuscript will already be made available publicly, so that anyone can make comments.)

%\todostefan{added commenting}

\subsection{Revised typesetting}

If necessary authors may revise their text taking into account the comments from the community proofreading stage.

\subsection{Final check}

Series Editors and Press Coordinators do a final check. If further changes are necessary, the
typesetting is adjusted again.


\subsection{Publication}

Once author, Series Editors AND Press Coordinators have given their imprimatur, the book is published by the Press Coordinators.


\section{Editors' information}
\label{sec-editors-information}

There will be two newsletters per month to inform all series editors about new submissions, accepted
manuscripts, published books and other news. 

\include{lsp-authors}
\include{lsp-latex}
\include{lsp-publication}


\backmatter

\bibliography{lsp-abbrev,lsp-guidelines}

%\cleardoublepage

\clearpage
\pdfbookmark[0]{Index}{Index}
\pdfbookmark[1]{Expression index}{Expression index}
\printindex[wrd]
\pdfbookmark[1]{Reverse expression index}{Reverse expression index}
\printindex[rwrd]
\pdfbookmark[1]{Name index}{Name index}
\printindex[aut]
\pdfbookmark[1]{Language index}{Language index}
\printindex[lan]
\pdfbookmark[1]{Subject index}{Subject index}
\printindex

                              
\end{document}
      
%      <!-- Local IspellDict: en_US-w_accents -->



On 19/03/2014 15:24, Sebastian Nordhoff wrote:
> da kann man noch ganz viele Sachen eintragen, zB Zielgruppe, Rechte etc
> (derzeit alles blank). Gibt es dafür schon einen Prozess? Sollen die
> Autoren hier Vorschläge machen, die Herausgeber, oder die Techniker (aka
> SN)?
Autoren machen Vorschläge, Reihenherausgeber segnen ab, würde ich sagen.
> Ich würde mal vorschlagen, ich mache das jetzt für Corinna nach bestem
> Wissen und Gewissen, aber in Zukunft sollten diese Dinge einheitlich
> geregelt sein
Genau, das muss ein Punkt in den Guidelines werden.

Aber wo genau erscheint diese Info?

M.



Würde ich nicht so streng sehen, aber ich persönlich habe eine Präferenz für "The term accusative" (also kursiv). Dann kann man die allgemeinen Regeln formulieren:

– Kursiv für metasprachliche Verwendung
– doppelte Anführungszeichen für Distanzierung (was andere sagen: Zitate, und von anderen verwendete Termini)
– einfache Anführungszeichen für Bedeutungen

(Stefan: Mir ist unklar, ob etwas davon in die allgemeinen LangSci-Guidelines rein soll. Bei manchen
dieser Dinge gibt es einfach unterschiedliche Konventionen.)


> Machst Du das weiter? Ich würde sonst alles einfach so an Corinna
> forwärtsen.
Hab gerade an Corinna geschrieben. Stefan, du kannst entscheiden, welche von diesen Dingen noch in
> die Guidelines rein sollen (eventuell, dass Termini kursiv sein sollen?, oder dass nach e.g. kein
> Komme kommt?). 

FAQ:

Ich vermute, dass Du \mainmatter nicht verwendet hast.

Im github gibt es ein Repository lsp-books, wo die Bücher zur Endredaktion reinkommen. Das ist öffentlich, allerdings nicht googlebar. Wenn Du willst, kann ich Dein Werk schon mal reinstellen. Dann haben alle Zugriff.


