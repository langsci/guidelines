\documentclass[output=paper]{langscibook}
\author{Sebastian Nordhoff}
\title{langsci-gb4e}

\usepackage{langsci-gb4e}
\usepackage{langsci-optional}
\usepackage{listings}
\lstset{  
    basicstyle=\fontfamily{pcr}\selectfont\footnotesize\color{blue},
    keywordstyle=\color{black}\bfseries, % style for keywords
    numbers=none, % where to put the line-numbers
    numberstyle=\tiny, % the size of the fonts that are used for the line-numbers     
    backgroundcolor=\color{gray},
    showspaces=false, % show spaces adding particular underscores
    showstringspaces=false, % underline spaces within strings
    showtabs=false, % show tabs within strings adding particular underscores
    frame=single, % adds a frame around the code
    tabsize=2, % sets default tabsize to 2 spaces
    rulesepcolor=\color{gray},
    rulecolor=\color{black},
    captionpos=b, % sets the caption-position to bottom
    breaklines=true, % sets automatic line breaking
    breakatwhitespace=false, 
}
\newcommand{\cmd}[1]{\texttt{\textbackslash#1}}
\newcommand{\env}[1]{\texttt{#1}}
\begin{document}
  \maketitle

\section{Introduction}
This document describes the \texttt{langsci-gb4e} package for typesetting linguistic examples. It builds upon the popular \texttt{gb4e} (by a) and \texttt{cgloss} (by X) packages. It also includes the package \texttt{jambox} by Alexis Dimitriadis. 
\section{History}
\section{Standard usage}
This manual starts with the most common cases and describes the foundations and the more complicated cases later. 

To produce a standard example, use \cmd{ea} before and \cmd{z} after
 
\begin{minipage}{.45\textwidth}
\begin{lstlisting}
  \ea 
  The cat is on the mat
  \z
\end{lstlisting}
\end{minipage}
\parbox[b]{.45\textwidth}{
  \ea 
  The cat is on the mat
  \z
  }

\section{Judgments}  
  To add judgments, there is a quick and dirty way and a proper way. 

\subsection{Quick and dirty way}  
Simply add a \texttt{*} in front of the sentence (or any other judgment). In groups of examples, this will look bad because vertical alignment is off (\ref{ex:judg1}--\ref{ex:judg2}).

\begin{minipage}{.45\textwidth}
\begin{lstlisting}
  \ea 
  The cat is on the mat
  \z
  \ea 
  * The cat are on the mat
  \z
\end{lstlisting}
\end{minipage}
\parbox[b]{.45\textwidth}{
  \ea\label{ex:judg1} 
  The cat is on the mat
  \z
  \ea\label{ex:judg2} 
  * The cat are on the mat
  \z
  }
  
\subsection{Proper way}  
The proper way puts the judgment between \texttt{[]} and does the same for empty judgments in a group. The sentence itself is put in \texttt{\{\}}. In this way, the examples align nicely. (\ref{ex:judg3}--\ref{ex:judg4}).
  
\begin{minipage}{.45\textwidth}
\begin{lstlisting}
  \ea[]{ 
  The cat is on the mat
  }
  \z 
  \ea[*]{ 
  The cat are on the mat
  }
  \z
\end{lstlisting}
\end{minipage}
\parbox[b]{.45\textwidth}{
  \ea[]{\label{ex:judg3}
  The cat is on the mat
  }
  \z 
  \ea[*]{\label{ex:judg4}
  The cat are on the mat
  }
  \z
  }
  
 
\subsection{Lists of examples}
If there are several examples in a row, you can use only one \cmd{z} at the very end and use \cmd{ex} instead of \cmd{ea} for examples after the first one (\ref{ex:ex:second}--\ref{ex:ex:end}).

\begin{minipage}{.45\textwidth}
\begin{lstlisting}
\ea I like the flowers
\ex I like the daffodils
\ex I like the mountains
\ex I like the rolling hills
\z
\end{lstlisting}
\end{minipage}
\parbox{.45\textwidth}{
\ea I like the flowers\label{ex:ex:start}
\ex I like the daffodils\label{ex:ex:second}
\ex I like the mountains
\ex I like the rolling hills\label{ex:ex:end}
\z
}

\subsection{Subexamples}
There are three predefined level of examples. \cmd{ea} opens a new level and prints the first identifier; \cmd{z} closes the last level. \cmd{ex} adds a further example but does not change levels. 

\begin{minipage}{.55\textwidth}
\begin{lstlisting}
\ea one
    \ea eins
    \ex una
    \z
\ex two
    \ea zwei
    \ex dos
    \z
\ex three
    \ea drei
        \ea who needs all these
        \ex levels of subexamples
        \z
    \z
\z
 \end{lstlisting}
\end{minipage}
\parbox{.45\textwidth}{
\ea one
    \ea eins
    \ex una
    \z
\ex two
    \ea zwei
    \ex dos
    \z
\ex three
    \ea drei
        \ea who needs all these
        \ex levels of subexamples
        \z
    \z
\z
}

\section{The environments \texttt{exe} and \texttt{xlist}}
The commands \cmd{ea} and \cmd{z} are shorthands for the environments \env{exe} (highest level) and \env{xlist} (subexamples and below). \cmd{ea} works like \verb+\begin{exe}\ex+ or \verb+\begin{xlist}\ex+, as the case may be. \cmd{z} works like \verb+\end{exe}+ or \verb+\end{xlist}+. In some cases, it can be necessary to resort to the environments instead of the shorthands, but this is rare. 


\begin{minipage}{.55\textwidth}
\begin{lstlisting}
\begin{exe}
    \ex one
    \begin{xlist}
        \ex eins
        \ex una
    \end{xlist}
    \ex two
    \begin{xlist}
        \ex zwei
        \ex dos
    \end{xlist}    
    \ex three
    \begin{xlist}
        \ex drei        
        \begin{xlist}
            \ex who needs all these
            \ex levels of subexamples
        \end{xlist}
    \end{xlist}
\end{exe}
 \end{lstlisting}
\end{minipage}
\parbox{.45\textwidth}{
\begin{exe}
    \ex one
    \begin{xlist}
        \ex eins
        \ex una
    \end{xlist}
    \ex two
    \begin{xlist}
        \ex zwei
        \ex dos
    \end{xlist}    
    \ex three
    \begin{xlist}
        \ex drei        
        \begin{xlist}
            \ex who needs all these
            \ex levels of subexamples
        \end{xlist}
    \end{xlist}
\end{exe}
}




\section{Advanced examples}
Sometimes, you want to have a particular identifier for a particular example. This can be achieved with \cmd{exi}.

\ea Normal example
\exi{(0)} Particular example
\ex Normal example
\z

\subsection{exi}
\subsection{primes}
\subsection{crossrefs}
\subsection{repeated}
\section{glossing}
\subsection{gll}
\subsection{glll}
\subsection{glt}
\subsection{~}
\subsection{\ O}
\section{series}
\section{lgr}
\section{jambox}
\section{what's removed}
\subsection{Xbar}
\subsection{arrows}
\subsection{greek}
\end{document}
