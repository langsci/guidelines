%% -*- coding:utf-8 -*-
\documentclass[ number=??
                ,series=lnls,
                ,isbn=000-0-000000-00-0,
                ,url=http://langsci-press.org/catalog/book/0,
	        ,output=long    % long|short|inprep              
	        %,blackandwhite
	        %,smallfont
	        ,draftmode  
		  ]{LSP/langsci}                          
\author{Stefan Müller and Sebastian Nordhoff}
\title{Guidelines for e}
\begin{document}
\chapter{Guidelines for editors}

\section{Decision structure}


Each \lsp series has a team of Series Editors, who decide which books are accepted for the
series. There can be up to three Series Editors per series; if more people are involved at the top
level, one or two have to be the Chief Editors, and the others are Consulting Editors (or simply
Editors).

In addition, each book series normally has an Editoral Board of 10--35 members. The Editorial Board
members advise the Series Editors in various ways concerning the series, in particular by writing
manuscript reviews. However, the list of names of the Editoral Board also serves to indicate the
kind of orientation that the series is inteded to take, and not least to give prestige to the
series. Editorial Board membership is normally for a period of three years (renewable).

For the first seven books in each series, acceptance is conditional on approval by the Press
Coordinators. This ensures that there is agreement between the Series Editors and the Press
Coordinators on the level of quality of the series. This is important to ensure a uniformly high
quality of all series.

\section{Series web pages}

Each series has a homepage, which lists the Series Editors, the Editorial Board members (with
affiliation), and contains an Aims and Scope statement.

All published books are listed on the page of the series. (They can also be found elsewhere on the
\lsp site, e.g. under ``\href{http://langsci-press.org/catalog}{Catalog}''.) This page may also list
forthcoming books, i.e. books which have been accepted, revised and approved and are at the
production stage.

As soon as a book has been accepted and approved, it can be put on the website as ``forthcoming'',
with the bibliographical information, but without the actual downloadable file. This will serve the
purpose of advance publicity.

(more details will follow later)

\section{Types of book manuscripts}

Language Science Press books may be monographs or edited volumes in English, German, French, Spanish,
and Portugese. Which languages are accepted depends on the particular series.

The manuscripts should have a size of at least 80 pages and at most 800 pages.
% We decided to remove the upper limit. IDS grammar has 2500 pages, what about dictionaries?
There are no technical reasons for excluding shorter and longer manuscripts, but such manuscripts
are not clearly within the scope of what readers would expect when they hear ``book''. Shorter works
are perhaps better published as journal articles, and longer works are difficult to organize a
serious reviewing process for.

(more details will follow later)


\section{Submission and reviewing procedure (monographs)}

Book manuscripts are officially submitted by entering them into the OMP system. Of course,
informal preliminary submission (by e-mail or by some file sharing mechanism) is
possible. Official submission implies that all Series Editors (as well as the Press Coordinators)
are informed of the submission, if the submission is done without OMP.


In a next step, the manuscript is made available to the reviewers via the OMP system (initially,
while not everyone is familiar with it, this can be done informally, e.g. by e-mail). For each book
manuscript, at least two reviews are solicited, within a time frame of two months. The reviews are
made available to the Press Coordinators. The Series Editors may override the recommendations of the
reviewers, but if all reviewers are mostly negative, this needs to be justified to the Press
Coordinators.

If a reviewer does not react even after three months, it is recommended that the Series Editors
solicit at least one additional review. If within six months after submission fewer than two reviews
are returned, the manuscript counts as rejected.

If a manuscript was rejected, the same author may submit another manuscript a year after the
submission of the rejected manuscript. The new manuscript may be similar to the originally submitted
manuscript, so the author may think of this as a ``resubmission''. However, there is no official
resubmission procedure in Language Science Press, and there is no ``revise and resubmit'' decision.

Note that Language Science Press does not issue ``contracts'' on the basis of book proposals, like
other publishers do. Book proposals may be discussed informally with the Series Editors, and the
Editors may informally encourage the author to submit a book on the basis of an informal book
proposal, but none of this has any binding status.

\section{Submission and reviewing procedure (edited volumes)}

For edited volumes, the Series Editors may adopt the same procedure as for monographs, or
alternatively they may accept the volume without review, i.e. they delegate the quality control to
the book editor. However, this is possible only if the papers underwent a comments \& revision
process, and if upon submission, the book editor gives a full account of the comments \& revision
procedure to the Series Editors. In such a case, a book manuscript may be accepted without revision.

\section{Acceptance}

On the basis of the reviewers' reports, the Series Editors decide whether the book is accepted for
the series or not.

If revisions are needed or recommended (as is likely to be the case), then this is a preliminary
acceptance, conditional on proper execution of revisions. However, peliminary acceptance means that
an author is allowed to cite the book as ``to appear with Language Science Press''.

Upon acceptance of a book manuscript, not only the author and the Press Coordinators, but also all
the other Series Editors are informed, so that they stay informed of developments within the entire
Press (see Section~\ref{sec-editors-information}).
%\todostefan{This should be done by OMP.}


\section{Revision}

If a book manuscript is accepted, the Series Editors convey the reviews and their own comments to
the author, and the author is asked to revise the manuscript.

The Series Editors may specify some Required Changes on which the definitive acceptance is
conditional. The Required Changes may only be highly specific changes that are not very
time-consuming. Vague proposals for changes (``the approach needs to be more firmly grounded in
theory'', etc.), or changes that require a lot of additional work, are not acceptable as Required Changes.

Apart from the Required Changes, authors may choose to ignore recommended changes, but these cases
need to be justified to the Series Editors. In the case of a serious disagreement between author and
Series Editors, the Press Coordinators are ready to mediate.

If the changes are made as requested the book will receive Definitive Acceptance.  The revision
stage includes proofreading. Like the revision of the content, this is the Series Editors'
responsibility, but the \lsp Community will be able to help with this. (Details will follow later.)

\section{Production}

Once the revised version of a manuscript has been returned by the author and Definitively Accepted
by the Series Editors, production can begin.

\subsection{Rough typesetting}

LaTeX styles are applied, figures are created in the proper way, etc.

\subsection{Formal contract}

At this stage, the author signs a contract with the FU Berlin (which is reponsible for hosting and permanent archiving) about the legal publication of the book. The contract form can be downloaded from the following page:

http://edocs.fu-berlin.de/docs/content/main/autoren/vertraege.xml?lang=en

Basically only the author's address needs to be filled in, as well as the book title and the URL (http://langsci-press.org/catalog/book/...).

This contract is necessary for the application for an ISBN number, which is needed for typesetting.

\subsection{Metadata and catalog}

The Series Editors/Authors enter the following metadata about the book into OMP:
\begin{itemize}
\item book synopsis (for the web page and back cover)
\item author bio
\item add keywords, regions, languages, and so on
\end{itemize}
The book also needs to be assigned to a category. At the moment, we are working with the following
categories:
\begin{itemize}
\item Phonetics and Phonology Phonetics
\begin{itemize}
\item Phonetics
\item Phonology
\end{itemize}
\item Morphology
\item Syntax
\item Semantics
\item Pragmatics
\item Historical Linguistics
\begin{itemize}
\item Comparative Historical Linguistics 
\end{itemize}
\item Typology
\end{itemize}
Once all these things have been taken care of, the book can be announced in the catalog as ``forthcoming''.

\subsection{Community proofreading/commenting}

(Details will follow later. Maybe at this stage the manuscript will already be made available publicly, so that anyone can make comments.)

%\todostefan{added commenting}

\subsection{Revised typesetting}

If necessary authors may revise their text taking into account the comments from the community proofreading stage.

\subsection{Final check}

Series Editors and Press Coordinators do a final check. If further changes are necessary, the
typesetting is adjusted again.


\subsection{Publication}

Once author, Series Editors AND Press Coordinators have given their imprimatur, the book is published by the Press Coordinators.


\section{Editors' information}
\label{sec-editors-information}

There will be two newsletters per month to inform all series editors about new submissions, accepted
manuscripts, published books and other news. 
\end{document}